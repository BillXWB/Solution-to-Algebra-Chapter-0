% Compile with PdfLaTeX

\documentclass[12pt,letterpaper,boxed]{hmcpset}

% set 1-inch margins in the document
\usepackage[margin=1in]{geometry}

% include this if you want to import graphics files with /includegraphics

\usepackage{graphicx}
\usepackage{amssymb}
\usepackage[all]{xy}

\usepackage[usenames,dvipsnames]{color}
\usepackage[colorlinks,linkcolor=NavyBlue,anchorcolor=red,citecolor=green]{hyperref}

\usepackage{mathrsfs}
\usepackage{cancel}
\usepackage{cite}
\usepackage{setspace}

\renewcommand{\baselinestretch}{1.05} 

\renewcommand\appendix{\setcounter{secnumdepth}{-2}}

\newcommand{\Obj}{\mathrm{Obj}} 
\newcommand{\Hom}{\mathrm{Hom}} 
\newcommand{\GL}{\mathrm{GL}} 
\newcommand{\SL}{\mathrm{SL}} 
\newcommand{\Aut}{\mathrm{Aut}}
\newcommand{\Inn}{\mathrm{Inn}}
\newcommand{\Grp}{\mathsf{Grp}} 
\newcommand{\Set}{\mathsf{Set}}
\newcommand{\Ab}{\mathsf{Ab}}  
\newcommand{\R}{\mathbb{R}} 
\newcommand{\C}{\mathbb{C}}
\newcommand{\Q}{\mathbb{Q}}
\newcommand{\Z}{\mathbb{Z}}
\newcommand{\N}{\mathbb{N}}

% info for header block in upper right hand corner
\name{Huyi Chen}
\updatedate{2018/03/31}

\begin{document}
		
\problemlist{\textbf{Algebra, Chapter 0}\\By Paolo Aluffi}

\tableofcontents
\appendix

\section{Chapter I.\quad Preliminaries: Set theory and categories}

\subsection{\textsection1. Naive Set Theory}

\begin{problem}[1.6]
Define a relation $\sim$ on the set $\mathbb{R}$ of real numbers, by setting $a\sim b\iff b-a \in\mathbb{Z}$. Prove that this is an equivalence relation, and find a \textquoteleft compelling' description for $\mathbb{R}/\sim$. Do the same for the relation $\approx$ on the plane $\mathbb{R}\times\mathbb{R}$ defined by declaring $(a_1, a_2)\approx(b_1, b_2)\iff b_1-a_1 \in\mathbb{Z}$ and $b_2-a_2 \in\mathbb{Z}$. [\textsection II.8.1, II.8.10]
\end{problem}
\begin{solution}
Imaginatively, $\mathbb{R}/\sim$ can be viewed as a ring of length 1 by bending the real line $\mathbb{R}$.	Then we can rotate a ring around an axis of rotation to get $\mathbb{R}\times\mathbb{R}/\approx$, which makes a torus.
\end{solution}	




\subsection{\textsection2. Functions between sets}

\begin{problem}[2.1]
How many different bijections are there between a set $S$ with $n$ elements
and itself? [\textsection II.2.1]
\end{problem}
\begin{solution}
	There are $n!$ different bijections $S\rightarrow S$.
\end{solution}


\subsection{\textsection3. Categories}
\begin{problem}[3.1]
	Let $\mathsf{C}$ be a category. Consider a structure $\mathsf{C}^{op}$ with:
	\begin{itemize}
	\item $\Obj(\mathsf{C}^{op}) := \Obj(\mathsf{C})$;
	\item for $A$, $B$ objects of $\mathsf{C}^{op}$ (hence, objects of $\mathsf{C}$), $\Hom_{\mathsf{C}^{op}} (A,B) := \Hom_\mathsf{C}(B,A)$
	\end{itemize}
	Show how to make this into a category (that is, define composition of morphisms
	in $\mathsf{C}^{op}$ and verify the properties listed in \textsection3.1).
	Intuitively, the 'opposite' category $\mathsf{C}^{op}$ is simply obtained by 'reversing all the
	arrows' in C. [5.1, \textsection VIII.1.1, \textsection IX.1.2, IX.1.10]
\end{problem}
\begin{solution}
	\begin{itemize}
		\item For every object $A$ of $\mathsf{C}$, there exists one identity morphism $1_A\in\Hom_\mathsf{C}(A,A)$. Since $\Obj(\mathsf{C}^{op}) := \Obj(\mathsf{C})$ and $\Hom_{\mathsf{C}^{op}} (A,A) := \Hom_\mathsf{C}(A,A)$, for every object $A$ of $\mathsf{C}^{op}$, the identity on $A$ coincides with $1_A\in\mathsf{C}$. 
		\item For $A$, $B$, $C$ objects of $\mathsf{C}^{op}$ and $f\in\Hom_{\mathsf{C}^{op}} (A,B)=\Hom_\mathsf{C}(B,A)$, $g\in\Hom_{\mathsf{C}^{op}} (B,C)=\Hom_\mathsf{C}(C,B)$, the composition laws in $\mathsf{C}$ determines a morphism $f*g$ in $\Hom_{\mathsf{C}} (C,A)$, which deduces the composition defined on $\mathsf{C}^{op}$:
		\[
		\begin{aligned}
		\Hom_{\mathsf{C}^{op}} (A,B)\times\Hom_{\mathsf{C}^{op}} (B,C)&\longrightarrow \Hom_{\mathsf{C}^{op}} (A,C)\\
		(f,g)&\longmapsto g\circ f:=f*g
		\end{aligned}
		\]
		\item Associativity. If $f\in\Hom_{\mathsf{C}^{op}} (A,B)$, $g\in\Hom_{\mathsf{C}^{op}} (B,C)$, $h\in\Hom_{\mathsf{C}^{op}} (C,D)$, then
		\[
		f\circ(g\circ h)=f\circ(h*g)=(h*g)*f=h*(g*f)=(g*f)\circ h=(f\circ g)\circ h.
		\]
		\item Identity. For all $f\in\Hom_{\mathsf{C}^{op}} (A,B)$, we have
		\[
		f\circ 1_A=1_A*f=f,\quad 1_B\circ f=f*1_B=f.
		\]
	\end{itemize}
	Thus we get the full construction of $\mathsf{C}^{op}$.
\end{solution}

\subsection{\textsection4. Morphisms}
\begin{problem}[4.2]
	In Example 3.3 we have seen how to construct a category from a set endowed	with a relation, provided this latter is reflexive and transitive. For what types of relations is the corresponding category a groupoid (cf. Example 4.6)? [\textsection4.1]
\end{problem}
\begin{solution}
    For a reflexive and transitive relation $\sim$ on a set $S$, define the category $\mathsf{C}$ as follows:
    \begin{itemize}
    	\item Objects: $\mathrm{Obj}(\mathsf{C})=S$;
    	\item Morphisms: if $a, b$ are objects (that is: if $a, b \in S$) then let 
    	\[
    	\mathrm{Hom}_\mathsf{C}(a, b)=
    	\left\{
    	\begin{aligned}
    	&(a, b)\in S\times S &\text{ if } a\sim b\\	
    	& \emptyset &\text{ otherwise}\\
    	\end{aligned}
    	\right.
    	\]
    \end{itemize}
    In Example 3.3 we have shown the category. If the relation $\sim$ is endowed with symmetry, we have
    \[
    (a,b)\in\mathrm{Hom}_\mathsf{C}(a, b)\implies a\sim b\implies b\sim a\implies (b,a)\in\mathrm{Hom}_\mathsf{C}(b, a).
    \]
    Since
    \[
    (a,b)(b, a)=(a,a)=1_a,\quad(b, a)(a,b)=(b,b)=1_b,
    \]
    in fact $(a,b)$ is an isomorphism. From the arbitrariness of the choice of $(a,b)$, we show that $\mathsf{C}$ is a groupoid. Conversely, if $\mathsf{C}$ is a groupoid, we can show the relation $\sim$ is symmetric. To sum up, the category $\mathsf{C}$ is a groupoid
    if and only if the corresponding relation $\sim$ is an equivalence relation.
\end{solution}

\subsection{\textsection5. Universal properties}
\begin{problem}[5.1]
	Prove that a final object in a category $\mathsf{C}$ is initial in the opposite category $\mathsf{C}_{op}$	(cf. Exercise 3.1).
\end{problem}
\begin{solution}
	An object $F$ of $\mathsf{C}$ is final in $\mathsf{C}$ if and only if
	\[
	\forall A \in \Obj(\mathsf{C}) : \Hom_\mathsf{C}(A,F) \text{ is a singleton.}
	\]
	That is equivalent to
	\[
	\forall A \in \Obj(\mathsf{C}_{op}) : \Hom_{\mathsf{C}_{op}}(F,A) \text{ is a singleton,}
	\]
	which means $F$ is initial in the opposite category $\mathsf{C}_{op}$.	
\end{solution}

\section{Chapter II.\quad Groups, first encounter}

\subsection{\textsection1. Definition of group}
\begin{problem}[1.1]
	Write a careful proof that every group is the group of isomorphisms of a groupoid. In particular, every group is the group of automorphisms of some object in some category.
\end{problem}
\begin{solution}
	Assume $G$ is a group. Define a category $\mathsf{C}$ as follows:
	\begin{itemize}
		\item Objects: $\mathrm{Obj}(\mathsf{C})=\{*\}$;
		\item Morphisms: $\mathrm{Hom}_\mathsf{C}(*,*)=\mathrm{End}_\mathsf{C}(*)=G$.
	\end{itemize}	
	The composition of homomorphism is corresponding to the multiplication between two elements in $G$. The identity morphism on $*$ is $1_*=e_G$, which satisfies for all $g\in \mathrm{Hom}_\mathsf{C}(*,*)$,
	\[
	ge_G=e_Gg=g,
	\]
	and
	\[
	gg^{-1}=e_G,\;g^{-1}g=e_G.
	\]
	Thus any homomorphism $g\in \mathrm{Hom}_\mathsf{C}(*,*)$ is an isomorphism and accordingly $\mathsf{C}$ is a groupoid. Now we see $G=\mathrm{End}_\mathsf{C}(*)$ is the group of isomorphisms of a groupoid. Moreover, supposing that $*$ is an object in some category $\mathsf{D}$, $G$ would be the group of automorphisms of $*$, which is denoted as $\mathrm{Aut}_\mathsf{D}(*)$.
\end{solution}

\begin{problem}[1.4]
	Suppose that $g^2 = e$ for all elements $g$ of a group $G$; prove that $G$ is commutative.
\end{problem}
\begin{solution}
	For all $a,b\in G$,
	\[
	abab=e\implies a(abab)b=ab\implies (aa)ba(bb)=ab\implies ba=ab.
	\]
\end{solution}

\subsection{\textsection2. Examples of groups}
\begin{problem}[2.1]
	One can associate an $n\times n$ matrix $M_\sigma$ with a permutation $\sigma \in S_n$, by
	letting the entry at $(i, \sigma(i))$ be 1, and letting all other entries be 0. For example,
	the matrix corresponding to the permutation
	\[
	\sigma=\left(
	\begin{matrix}	
	1 & 2 & 3\\	
    3 & 1 & 2	
	\end{matrix}
	\right)\in S_3
	\]
	would be 
	\[
	M_\sigma=\left(
	\begin{matrix}	
	0 & 0 & 1\\	
	1 & 0 & 0\\	
	0 & 1 & 0
	\end{matrix}
	\right)
	\]
	Prove that, with this notation,
	\[
	M_{\sigma\tau}=M_{\sigma}M_{\tau}
	\]
	for all $\sigma,\tau\in S_n$, where the product on the right is the ordinary product of matrices.
\end{problem}
\begin{solution}
	By introducing the Kronecker delta function	
    \[
    \delta _{{i,j}}=
    	\begin{cases}
    	0{\text{\quad if }}i\neq j,\\
    	1{\text{\quad if }}i=j,
    	\end{cases}
    \]
	the entry at $(i,j)$ of the matrix $M_{\sigma\tau}$ can be written as
	\[
	(M_{\sigma\tau})_{i,j}=\delta_{\tau(\sigma(i)),j}
	\]
	and the entry at $(i,j)$ of the matrix $M_{\sigma}M_{\tau}$ can be written as
	\[
	(M_{\sigma}M_{\tau})_{i,j}=\sum_{k=1}^{n}(M_{\sigma})_{i,k}(M_{\tau})_{k,j}=\sum_{k=1}^{n}\delta_{\sigma(i),k}\cdot\delta_{\tau(k),j}=\sum_{k=1}^{n}\delta_{\sigma(i),k}\cdot\delta_{k,\tau^{-1}(j)}=\delta_{\sigma(i),\tau^{-1}(j)},
	\]
	where the last but one equality holds by the fact 
	\[
	\tau(k)=j\iff k=\tau^{-1}(j).
	\]
	Noticing that
	\[
	\tau(\sigma(i))=j\iff \sigma(i)=\tau^{-1}(j),
	\]
    we see $M_{\sigma\tau}=M_{\sigma}M_{\tau}$
    for all $\sigma,\tau\in S_n$.
\end{solution}


\begin{problem}[2.2]
	Prove that if $d \le n$, then $S_n$ contains elements of order $d$.
\end{problem}
\begin{solution}
	The cyclic permutation 
	\[
	\sigma=(1\;2\;3\cdots d)
	\]
    is an element of order $d$ in $S_n$.
\end{solution}



\begin{problem}[2.3]
	For every positive integer $n$ find an element of order $n$ in $S_\mathbb{N}$.
\end{problem}
\begin{solution}
The cyclic permutation 
\[
\sigma=(1\;2\;3\cdots n)
\]
is an element of order $d$ in $S_n$.
\end{solution}



\begin{problem}[2.4]	
	Define a homomorphism $D_8 \rightarrow S_4$ by labeling vertices of a square, as we did
	for a triangle in \textsection2.2. List the 8 permutations in the image of this homomorphism.
\end{problem}
\begin{solution}
	The image of $n$ rotations under the homomorphism are
	\[
	\sigma_1=e_{D_8},\;\sigma_2=(1\;2\;3\;4),\;\sigma_3=(1\;3)(2\;4),\;\sigma_4=(1\;4\;3\;2).
	\]
	The image of $n$ reflections under the homomorphism are
	\[
	\sigma_5=(1\;3),\;\sigma_6=(2\;4),\;\sigma_7=(1\;2)(3\;4),\;\sigma_8=(1\;4)(3\;2).
	\]
\end{solution}



\begin{problem}[2.11]	
	Prove that the square of every odd integer is congruent to 1 modulo 8.
\end{problem}
\begin{solution}
	Given an odd integer $2k+1$, we have
	\[
	(2k+1)^2=4k(k+1)+1,
	\]
	where $k(k+1)$ is an even integer. So $(2k+1)^2\equiv1\mod 8$.
\end{solution}



\begin{problem}[2.12]	
	Prove that there are no integers $a, b, c$ such that $a^2+b^2=3c^2$. (Hint: studying the equation $[a]^2_4+[b]^2_4=3[c]^2_4$ in $\mathbb{Z}/4\mathbb{Z}$, show that $a, b, c$ would all have to be even. Letting $a=2k, b=2l,c=2m$, you would have $k^2+l^2=3m^2$. What's wrong with that?)
\end{problem}
\begin{solution}
	\[
	a^2+b^2=3c^2\implies [a]^2_4+[b]^2_4=3[c]_4^2.
	\]
	Noting that $[0]^2_4=[0]_4,[1]^2_4=[1]_4,[2]^2_4=[0]_4,[3]^2_4=[1]_4$, we see $[c]_4^2$ must be $[0]_4$ and so do $[a]_4^2$ and $[b]_4^2$. Hence  $[a]_4,[b]_4,[b]_4$ can only be $[0]_4$ or $[2]_4$, which justifies letting $a=2k_1, b=2l_2,c=2m_1$. After substitution we have $k^2+l^2=3m^2$. Repeating this process $n$ times yields $a=2^nk_n, b=2^nl_n,c=2^nm_n$. For a sufficiently large number $N$, the absolute value of $k_N,l_N,m_N$ must be less than 1. Thus we conclude that $a=b=c=0$ is the unique solution to the equation $a^2+b^2=3c^2$.
\end{solution}
    


\begin{problem}[2.13]	
	Prove that if $\gcd(m, n) = 1$, then there exist integers $a$ and $b$ such that $am + bn = 1$. (Use Corollary 2.5.) Conversely, prove that if $am+ bn = 1$ for some integers $a$ and $b$, then $\gcd(m, n) = 1$. [2.15, \textsection V.2.1, V.2.4]
\end{problem}
\begin{solution}
	Applying corollary 2.5, we have $\gcd(m, n) = 1$ if and only if $[m]_n$ generates $\mathbb{Z}/n\mathbb{Z}$. Hence
	\[
	\gcd(m, n) = 1\iff a[m]_n=[1]_n\iff [am]_n=[1]_n\iff am+ bn = 1.
	\]
\end{solution}
    


\begin{problem}[2.15]	
	Let $n > 0$ be an odd integer.
	\begin{itemize}
		\item Prove that if $\gcd(m, n) = 1$, then $\gcd(2m+ n, 2n) = 1$. (Use Exercise 2.13.)
		\item Prove that if $\gcd(r, 2n) = 1$, then $\gcd(\frac{r+n}{2}, n) = 1$. (Ditto.)		
		\item Conclude that the function $[m]_n\rightarrow[2m + n]_{2n}$ is a bijection between $(\mathbb{Z}/n\mathbb{Z})^*$ and $(\mathbb{Z}/2n\mathbb{Z})^*$. 
	\end{itemize}
	The number $\phi(n)$ of elements of $(\mathbb{Z}/n\mathbb{Z})^*$ is Euler’s $\phi(n)$-function. The reader has just	proved that if $n$ is odd, then $\phi(2n) = \phi(n)$. Much more general formulas will be given later on (cf. Exercise V.6.8). [VII.5.11]
\end{problem}
\begin{solution}
	\begin{itemize}
		\item According to Exercise 2.13,
		\[
		\gcd(m, n) = 1\implies am+bn=1\implies \frac{a}{2}(2m+n)+\left(b-\frac{a}{2}\right)n=1.
		\]
		If $a$ is even, we have shown $\gcd(2m+ n, 2n) = 1$. Otherwise we can let $a'=a+n$ be an even integer and $b'=b-m$. Then it holds that
		\[
		\frac{a'}{2}(2m+n)+\left(b'-\frac{a'}{2}\right)n=1,
		\]
		which also indicates $\gcd(2m+ n, 2n) = 1$.
		\item If $\gcd(r, 2n) = 1$, then $r$ must be an odd integer and accordingly
		\[
		\gcd(2r+2n, 4n) = 1\implies a(2r+2n)+b(4n)=1\implies 4a \frac{r+n}{2}+4bn=1,
		\]
		which is $\gcd(\frac{r+n}{2}, n) = 1$.
		\item It is easy to check that the function $f:(\mathbb{Z}/n\mathbb{Z})^*\rightarrow(\mathbb{Z}/2n\mathbb{Z})^*,\;[m]_n\mapsto[2m + n]_{2n}$ is well-defined. The fact
		\[
		\begin{aligned}
		f([m_1]_n)=f([m_2]_n)&\implies
		f([2m_1 + n]_{2n})=f([2m_2 + n]_{2n})\\
		&\implies (2m_1 + n)-(2m_2 + n)=2kn\\
		&\implies m_1-m_2=kn\\
		&\implies [m_1]_n=[m_2]_n
		\end{aligned}
		\]
		indicates that $f$ is injective. For any $[r]_{2n}\in(\mathbb{Z}/2n\mathbb{Z})^*$, we have
		\[
		\gcd(r, 2n) = 1\implies\gcd\left(\frac{r+n}{2},n\right) = 1\implies \left[\frac{r+n}{2}\right]_n\in(\mathbb{Z}/n\mathbb{Z})^*,
		\]
		and
		\[
		f\left(\left[\frac{r+n}{2}\right]_n\right)=[r+2n]_{2n}=[r]_{2n},
		\]
		which indicates that $f$ is surjective. Thus we show $f$ is a bijection.
	\end{itemize}
\end{solution}



\begin{problem}[2.16]	
	Find the last digit of $1238237^{18238456}$. (Work in $\mathbb{Z}/10\mathbb{Z}$.)
\end{problem}
\begin{solution}
	\[
	1238237^{18238456}\equiv7^{18238456}\equiv(7^4)^{4559614}\equiv2401^{4559614} \equiv 1 \mod 10,
	\]
	which indicates that the last digit of $1238237^{18238456}$ is 1.
\end{solution}




\begin{problem}[2.17]	
	Show that if $m\equiv m'\mod n$, then $\gcd(m,n) = 1$ if and only if $\gcd(m',n)=1$. [\textsection 2.3]
\end{problem}
\begin{solution}
	Assume that $m-m'=kn$. If $\gcd(m,n) = 1$, for any common divisor $d$ of $m'$ and $n$
	\[
	d|m',\;d|n\implies d|(m'+kn)\implies d|m\implies d=1,
	\]
	which means $\gcd(m',n)=1$. Likewise, we can show $\gcd(m',n) = 1\implies\gcd(m,n)=1$
\end{solution}


\subsection{\textsection3. The category $\mathsf{Grp}$} 

\begin{problem}[3.1]	
	Let $\varphi : G\rightarrow H$ be a morphism in a category $\mathsf{C}$ with products. Explain why
	there is a unique morphism
	\[
	(\varphi\times\varphi) : G \times G \longrightarrow H \times H .
	\]
	(This morphism is defined explicitly for $\mathsf{C} = \mathsf{Set}$ in \textsection 3.1.)
\end{problem}
\begin{solution}
	By the universal property of product in $\mathsf{C}$, there exist a unique morphism $(\varphi\times\varphi) : G \times G \longrightarrow H \times H$ such that the following diagram commutes.
    \[\xymatrix{
    	G\ar[r]^{\varphi} & H \\
    	G \times G\ar[u]^{\pi_G}\ar[d]_{\pi_G}\ar[r]^{\varphi\times\varphi} &  H\times H\ar[u]_{\pi_H}\ar[d]^{\pi_H} \\
    	G\ar[r]^{\varphi} & H 
    }\]
\end{solution}



\begin{problem}[3.2]	
	Let $\varphi : G\rightarrow H, \psi : H \rightarrow K$ be morphisms in a category with products, and
	consider morphisms between the products $G\times G, H\times H, K\times K$ as in Exercise 3.1.
	Prove that
	\[
	(\psi\varphi) \times(\psi\varphi)=(\psi \times \psi)(\varphi\times \varphi) .
	\]
	(This is part of the commutativity of the diagram displayed in \textsection 3.2.)
\end{problem}
\begin{solution}
	By the universal property of product in $\mathsf{C}$, there exists a unique morphism 
	\[
	(\psi\varphi) \times(\psi\varphi):G\times G\rightarrow K\times K
	\] 
	such that the following diagram commutes.
	\[\xymatrix{
		G\ar[rr]^{\psi\varphi} && H \\
		G \times G\ar[u]^{\pi_G}\ar[d]_{\pi_G}\ar[rr]^{	(\psi\varphi) \times(\psi\varphi)} &&  H\times H\ar[u]_{\pi_H}\ar[d]^{\pi_H} \\
		G\ar[rr]^{\psi\varphi} && H 
	}\]
    As the following commutative diagram tells us the composition 
    \[
    (\psi \times \psi)(\varphi\times \varphi):G\times G\rightarrow K\times K
    \]
    can make the above diagram commute,
	\[\xymatrix{
		G\ar[r]^{\varphi}\ar@/^1.6pc/[rr]^{\psi\varphi} & H\ar[r]^{\psi} & K \\
		G \times G\ar[u]^{\pi_G}\ar[d]_{\pi_G}\ar[r]^{\varphi\times\varphi} &  H\times H\ar[u]^{\pi_H}\ar[d]_{\pi_H}\ar[r]^{\psi\times\psi} &  K\times K \ar[u]^{\pi_K}\ar[d]_{\pi_K}\\
		G\ar[r]^{\varphi}\ar@/_1.6pc/[rr]_{\psi\varphi} & H \ar[r]^{\psi} & K
	}\]
    there must be $(\psi\varphi) \times(\psi\varphi)=(\psi \times \psi)(\varphi\times \varphi)$.
    
\end{solution}



\begin{problem}[3.3]	
	Show that if $G, H$ are abelian groups, then $G \times H$ satisfies the universal property for coproducts in $\mathsf{Ab}$.
\end{problem}
\begin{solution}
	Define two monomorphisms:
	\[
	i_G:G\longrightarrow G\times H,\;a\longmapsto (a,0_H)
	\]
	\[
	i_H:H\longrightarrow G\times H,\;b\longmapsto (0_G,b)
	\]
	We are to show that for any two homomorphisms $g:G\rightarrow M$ and $h:H\rightarrow M$ in $\mathsf{Ab}$, the mapping
	\[
	\begin{aligned}
	\varphi:\quad & G\times H\longrightarrow M,\\
	         & (a,b)\longmapsto g(a)+h(b)
	\end{aligned}
    \]
    is a homomorphism and makes the following diagram commute. 
	\[\xymatrix{
		G\ar[rd]^{g}\ar[d]_{i_G}  \\
		G \times H\ar[r]^{\varphi} &  M\\
		H\ar[ru]_{h}\ar[u]^{i_H}  
	}\]
	Exploiting the fact that $g,h$ are homomorphisms and $M$ is an abelian group, it is easy to check that $\varphi$ preserves the addition operation
	\[
	\begin{aligned}
	\varphi((a_1,b_1)+(a_2,b_2))&=\varphi((a_1+a_2,b_1+b_2))\\
	&=g(a_1+a_2)+h(b_1+b_2)\\
	&=(g(a_1)+g(a_2))+(h(b_1)+h(b_2))\\
	&=(g(a_1)+h(b_1))+(g(a_2)+h(b_2))\\
	&=g(a_1+b_1)+h(a_2+b_2)\\
	&=\varphi((a_1,b_1))+\varphi((a_2,b_2))
	\end{aligned}
	\]
	and the diagram commutes
	\[
	\varphi\circ i_G(a)=\varphi((a,0_H))=g(a)+h(0_H)=g(a)+0_M=g(a),
	\]
	\[
	\varphi\circ i_H(b)=\varphi((0_G,b))=g(0_G)+h(b)=0_M+h(b)=h(b).
	\]
	To show the uniqueness of the homomorphism $\varphi$ we have constructed, suppose a homomorphism $\varphi'$ can make the diagram commute. Then we have
	\[
	\varphi'((a,b))=\varphi'((a,0_H)+(0_G,b))=\varphi'(i_G(a))+\varphi'(i_H(b))=g(a)+h(b)=\varphi((a,b)),
	\]
	that is $\varphi'=\varphi$. Hence we show that there exist a unique homomorphism $\varphi$ such that the diagram commutes, which amounts to the universal property for coproducts in $\mathsf{Ab}$.
	
\end{solution}



\begin{problem}[3.4]	
	Let $G, H$ be groups, and assume that $G\cong H\times G$. Can you conclude that $H$ is trivial? (Hint: No. Can you construct a counterexample?)
\end{problem}
\begin{solution}
	Consider the function
	\[
	\begin{aligned}
	\varphi:\;&\mathbb{Z}\times\mathbb{Z}[x]\longrightarrow\mathbb{Z}[x]\\  
	  &(n,f(x))\longmapsto n+xf(x)
	\end{aligned}
	\]
	Firstly, we can show $\varphi$ is a homomorphism as follows
	\[
	\begin{aligned}
	\varphi((n_1,f_1(x))+(n_2,f_2(x)))&=\varphi((n_1+n_2,f_1(x)+f_2(x)))\\
	&=(n_1+n_2)+x(f_1(x)+f_2(x))\\
	&=(n_1+xf_1(x))+(n_2+xf_2(x))\\
	&=\varphi((n_1,f_1(x)))+\varphi((n_2,f_2(x))).
	\end{aligned}
	\]
	Secondly, we are to show $\varphi$ is a monomorphism. It follows by
	\[
	\varphi((n,f(x)))=n+xf(x)=0\implies n=0,\;f(x)=0\implies \ker\varphi=\{(0,0)\}.
	\]
	Lastly, since the cardinal numbers of both $\mathbb{Z}\times\mathbb{Z}[x]$ and $\mathbb{Z}[x]$ are $\aleph_0$, $\varphi$ is indeed an isomorphism. Therefore, as a counterexample we have $\mathbb{Z}[x]\cong\mathbb{Z}\times\mathbb{Z}[x]$.
\end{solution}



\begin{problem}[3.5]	
	Prove that $\mathbb{Q}$ is not the direct product of two nontrivial groups.
\end{problem}
\begin{solution}
    Consider the additive group of rationals $(\mathbb{Q},+)$. Assume that $\varphi$ is a isomorphism between the product $G\times H=\{(a,b)|a\in G,b\in H\}$ and $(\mathbb{Q},+)$. Note that $\{e_G\}\times H$ and $G\times \{e_H\}$ are subgroups in $G\times H$ and their intersection is the trivial group $\{(e_G,e_H)\}$. It is easy to check that bijection $\varphi$ satisfies $\varphi(A\cap B)=\varphi(A)\cap\varphi (B)$. So applying the fact we have
    \[
    \varphi(\{(e_G,e_H)\})=\varphi(\{e_G\}\times H\cap G\times \{e_H\})=\varphi(\{e_G\}\times H)\cap \varphi(G\times \{e_H\})=\{0\}.
    \] 
    Suppose both $\varphi(\{e_G\}\times H)$ and $\varphi(G\times \{e_H\})$ are nontrivial groups. If $\dfrac{p}{q}\in \varphi(\{e_G\}\times H)-\{0\}$ and $\dfrac{r}{s}\in \varphi(G\times \{e_H\})-\{0\}$, there must be 
    \[
    rp=rq\cdot\dfrac{p}{q}=ps\cdot\dfrac{r}{s}\in \varphi(\{e_G\}\times H)\cap\varphi(G\times \{e_H\}),
    \]
    which implies $rp=0$. Since both $\dfrac{p}{q}$ and $\dfrac{r}{s}$ are non-zero, it leads to a contradiction. Thus without loss of generality we can assume $\varphi(\{e_G\}\times H)$ is a trivial group $\{0\}$. 
    Since $\varphi$ is isomorphism, we see that for all $h\in H$,
    \[
    \varphi(e_G,h)=\varphi(e_G,e_H)=0\iff h=e_H.
    \]
    That is, $H$ is a trivial group. Therefore, we have shown $(\mathbb{Q},+)$ will never be isomorphic to the direct product of two nontrivial groups.
\end{solution}



\begin{problem}[3.6]	
	Consider the product of the cyclic groups $C_2,C_3$ (cf. \textsection 2.3): $C_2 \times C_3$. By Exercise 3.3, this group is a coproduct of $C_2$ and $C_3$ in $\mathsf{Ab}$. Show that it is not a coproduct of $C_2$ and $C_3$ in $\mathsf{Grp}$, as follows:
	\begin{itemize}
		\item find injective homomorphisms $C_2\rightarrow S_3$, $C_3 \rightarrow S_3$;
		\item arguing by contradiction, assume that $C_2\times C_3$ is a coproduct of $C_2, C_3$, and deduce that there would be a group homomorphism $C_2\times C_3\rightarrow S_3$ with certain properties;
		\item show that there is no such homomorphism.
	\end{itemize}
\end{problem}
\begin{solution}
	\begin{itemize}
	\item 
	Monomorphisms $g:C_2\rightarrow S_3$, $h:C_3 \rightarrow S_3$ can be constructed as follows:
	\[
	g([0]_2)=e,g([1]_2)=\left(
	\begin{matrix}	
	1 & 2 & 3\\	
	1 & 3 & 2	
	\end{matrix}
	\right).
	\]
	\[
	h([0]_3)=e,h([1]_3)=\left(
	\begin{matrix}	
	1 & 2 & 3\\	
	3 & 1 & 2	
	\end{matrix}
	\right),h([2]_3)=\left(
	\begin{matrix}	
	1 & 2 & 3\\	
	2 & 3 & 1	
	\end{matrix}
	\right).
	\]
	\item Supposing that $C_2\times C_3$ is a coproduct of $C_2, C_3$, there would be a unique group homomorphism $\varphi :C_2\times C_3\rightarrow S_3$ such that the following diagram commutes
	\[\xymatrix{
		C_2\ar[rd]^{g}\ar[d]_{i_{C_2}}  \\
		C_2 \times C_3\ar[r]^{\varphi} &  S_3\\
		C_3\ar[ru]_{h}\ar[u]^{i_{C_3}}  
	}\]
    In other words, for all $a\in C_2,b\in C_3$,
    \[
    \begin{aligned}
    \varphi(a,b)&=\varphi(([0]_2,b)+(a,[0]_3))=\varphi(([0]_2,b))\varphi((a,[0]_3))=\varphi(i_{C_3}(b))\varphi(i_{C_2}(a))=h(b)g(a)\\
    &=\varphi((a,[0]_3)+([0]_2,b))=\varphi((a,[0]_3))\varphi(([0]_2,b))=\varphi(i_{C_2}(a))\varphi(i_{C_3}(b))=g(a)h(b).
    \end{aligned}
    \]
	\item Since
	\[
	g([1]_2)h([1]_3)=\left(
	\begin{matrix}	
	1 & 2 & 3\\	
	1 & 3 & 2	
	\end{matrix}
	\right)
	\left(
	\begin{matrix}	
	1 & 2 & 3\\	
	3 & 1 & 2	
	\end{matrix}
	\right)=
	\left(
	\begin{matrix}	
	1 & 2 & 3\\	
	3 & 2 & 1	
	\end{matrix}
	\right),
	\]
	\[
	h([1]_3)g([1]_2)=
	\left(
	\begin{matrix}	
	1 & 2 & 3\\	
	3 & 1 & 2	
	\end{matrix}
	\right)
	\left(
	\begin{matrix}	
	1 & 2 & 3\\	
	1 & 3 & 2	
	\end{matrix}
	\right)
	=
	\left(
	\begin{matrix}	
	1 & 2 & 3\\	
	2 & 1 & 3	
	\end{matrix}
	\right),
	\]
	we see $g(a)h(b)\ne h(b)g(a)$ not always holds. The derived contradiction shows that $C_2\times C_3$ is not a coproduct of $C_2, C_3$ in $\mathsf{Grp}$. 
\end{itemize}	
\end{solution}

\begin{problem}[3.7]	
	Show that there is a surjective homomorphism $Z*Z\rightarrow C_2*C_3$. ($*$ denotes coproduct in $\mathsf{Grp}$.)
	
\end{problem}

\begin{solution}
	Consider the mapping
	\[
	\begin{aligned}
	\varphi:\;&\mathbb{Z}*\mathbb{Z}\longrightarrow C_2*C_3\\  
	&x^{m_1}y^{n_1}\cdots x^{m_k}y^{n_k}\longmapsto x^{[m_1]_2}y^{[n_1]_3}\cdots x^{[m_k]_2}y^{[n_k]_3}
	\end{aligned}
	\]
	Since
	\[
	\begin{aligned}
	&\varphi(x^{m_1}y^{n_1}\cdots x^{m_k}y^{n_k}x^{m_1'}y^{n_1'}\cdots x^{m_{k'}'}y^{n_{k}'})\\
	=&x^{[m_1]_2}y^{[n_1]_3}\cdots x^{[m_k]_2}y^{[n_k]_3}x^{[m_1']_2}y^{[n_1']_3}\cdots x^{[m_k']_2}y^{[n_k']_3}\\
	=&\varphi(x^{m_1}y^{n_1}\cdots x^{m_k}y^{n_k})\varphi(x^{m_1'}y^{n_1'}\cdots x^{m_{k'}'}y^{n_{k}'})
	\end{aligned},
	\]
	$\varphi$ is a homomorphism. It is clear that $\varphi$ is surjective. Thus we show there exists a surjective homomorphism $Z*Z\rightarrow C_2*C_3$.
\end{solution}

\begin{problem}[3.8]	
	Define a group $G$ with two generators $x, y$, subject (only) to the relations $x^2 = e_G, y^3 = e_G$. Prove that $G$ is a coproduct of $C_2$ and $C_3$ in $\Grp$. (The reader	will obtain an even more concrete description for $C_2*C_3$ in Exercise 9.14; it is called the modular group.) [\textsection3.4, 9.14]
	
\end{problem}

\begin{solution}
Given the maps $i_1:C_2\rightarrow G,[m]_2\mapsto x^m$ and $i_2:C_3\rightarrow G,[n]_3\mapsto y^n$, we can check that $i_1$, $i_2$ are homomorphisms. 
We are to show that for every group $H$ endowed with two homomorphisms $f_1:C_2\rightarrow H$, $f_2:C_3\rightarrow H$ , there would be a unique group homomorphism $\varphi :G\rightarrow H$ such that the following diagram commutes
\[\xymatrix{
	C_2\ar[rd]^{f_1}\ar[d]_{i_1}  \\
	G\ar[r]^{\varphi} &  H\\
	C_3\ar[ru]_{f_2}\ar[u]^{i_2}  
}\]	
or
\[
\varphi(i_1([m]_2))=\varphi(x^m)=\varphi(x)^m=f_1([m]_2),
\]
\[
\varphi(i_2([n]_3))=\varphi(y^n)=\varphi(y)^n=f_2([n]_3).
\]
Define $\phi:G\rightarrow H$ as $\phi(x^my^n)=f_1([m]_2)f_2([n]_3)$, $\phi(y^nx^m)=f_2([n]_3)f_1([m]_2)$. It is clear to see $\phi$ makes the diagram commute. Moreover, if $\varphi$ makes the diagram commute, it follows that for all $x^my^n,y^nx^m\in G$,
\[
\varphi(x^my^n)=\varphi(x^m)\varphi(y^n)=f_1([m]_2)f_2([n]_3),
\]
\[
\varphi(y^nx^m)=\varphi(y^n)\varphi(x^m)=f_2([n]_3)f_1([m]_2),
\]
which implies $\varphi=\phi$. Thus we can conclude $G$ is the coproduct of $C_2$ and $C_3$ in $\Grp$.

\end{solution}

\subsection{\textsection4. Group homomorphisms} 

\begin{problem}[4.1]
Check that the function $\pi_m^n$
defined in \textsection4.1 is well-defined, and makes the
diagram commute. Verify that it is a group homomorphism. Why is the hypothesis
$m | n$ necessary? [\textsection4.1]
\end{problem}

\begin{solution}
In \textsection4.1 the function $\pi_m^n$ is defined as
\[
\begin{aligned}
\pi_m^n\;: \mathbb{Z}/n\mathbb{Z} &\longrightarrow \mathbb{Z}/m\mathbb{Z}\\  
[a]_n&\longmapsto [a]_m
\end{aligned}
\]	
with the condition $m|n$. We can check that $\pi_m^n$ is well-defined as
\[
[a_1]_n=[a_2]_n\iff a_1-a_2=kn=(kl)m \implies[a_1]_m=[a_2]_m\iff\pi_m^n([a_1]_n)=\pi_m^n([a_2]_n).
\]	
Note $\pi_m^n(\pi_n(a))=\pi_m^n([a]_n)=[a]_m=\pi_m(a)$. The diagram in \textsection4.1 must commute.
\[\xymatrix{
	\mathbb{Z}\ar[rd]^{\pi_m}\ar[d]_{\pi_n}  \\
	\mathbb{Z}/n\mathbb{Z}\ar[r]_{\pi_m^n} &  \mathbb{Z}/m\mathbb{Z} 
}\]
Since
\[
\pi_m^n([a]_n+[b]_n)=[a+b]_m=[a]_m+[b]_m=\pi_m^n([a]_n)+\pi_m^n([b]_n),
\]
it follows that $\pi_m^n$ is a group homomorphism. Actually we have shown that without the hypothesis
$m|n$, $\pi_m^n$ may not be well-defined.
\end{solution}


\begin{problem}[4.2]
 Show that the homomorphism $\pi_2^4\times\pi_2^4:C_4\rightarrow C_2\times C_2$ is not an isomorphism. In fact, is there any nontrivial isomorphism $C_4\rightarrow C_2\times C_2$?
\end{problem}

\begin{solution}
	Let calculate the order of each non-zero element in both $C_4$ and $ C_2\times C_2$. For the group $C_4$,
	\[
	|[2]_4|=2,\quad\left|[1]_4\right|=\left|[3]_4\right|=4.
	\]
	For the group $C_2\times C_2$,
	\[
	|([1]_2,[0]_2)|=|([0]_2,[1]_2)|=|([1]_2,[1]_2)|=2.
	\]
	Since isomorphism must preserve the order, we can assert that there is no such isomorphism $C_4\rightarrow C_2\times C_2$.
\end{solution}


\begin{problem}[4.3]
	Prove that a group of order $n$ is isomorphic to $\mathbb{Z}/n\mathbb{Z}$ if and only if it contains
	an element of order $n$. [\textsection4.3]
\end{problem}

\begin{solution}
	Assume some group $G$ is isomorphic to $\mathbb{Z}/n\mathbb{Z}$. Since $|[1]_n|=n$ and isomorphism preserves the order, we can affirm that there is an element of order $n$ in $G$. 
	
	\noindent Conversely, assume there is a group $G$ of order $n$ in which $g$ is an element of order $n$. By definition we see $g^0,g^1,g^2\cdots g^{n-1}$ are distinct pairwise. Noticing group $G$ has exactly $n$ elements, $G$ must consist of $g^0,g^1,g^2\cdots g^{n-1}$. We can easily check that the function
	\[
	\begin{aligned}
	f\;: G&\longrightarrow \mathbb{Z}/n\mathbb{Z}\\  
	g^k&\longmapsto [k]_n
	\end{aligned}
	\]	
	is an isomorphism.	
\end{solution}


\begin{problem}[4.4]
	Prove that no two of the groups $(\mathbb{Z}, +)$, $(\mathbb{Q}, +)$, $(\mathbb{R},+)$ are isomorphic to one another. Can you decide whether $(\mathbb{R}, +)$, $(\mathbb{C}, +)$ are isomorphic to one another? (Cf. Exercise VI.1.1.)
\end{problem}

\begin{solution}
	Suppose there exists an isomorphism $f:\mathbb{Z}\rightarrow\mathbb{Q}$. Let $f(1)=p/q\ (p,q\in\mathbb{Z})$. If $p=1$, for all $n\in\mathbb{Z}$, we have
	\[
	f(n)=\frac{n}{q}\ne\frac{1}{2q}.
	\]
	If $p\ne1$, for all $n\in\mathbb{Z}$, we have
	\[
	f(n)=\frac{np}{q}\ne\frac{p+1}{q}.
	\]
	In both cases, it implies $f(\mathbb{Z})\nsubseteq\mathbb{Q}$.  Hence we see $f$ is not a surjection, which contradicts the fact that $f:\mathbb{Z}\rightarrow\mathbb{Q}$ is an isomorphism. Compare the cardinality of $\mathbb{Z}$, $\mathbb{Q}$, $\mathbb{R}$
	\[
	|\mathbb{Z}|=|\mathbb{Q}|<|\mathbb{R}|
	\] 
	and we show there exists no such isomorphisms like $f:\mathbb{Z}\rightarrow \mathbb{R}$ or $f:\mathbb{Q}\rightarrow \mathbb{R}$. 
	
	\noindent We can prove $(\mathbb{R}, +)$, $(\mathbb{C}, +)$ are isomorphic, if considering the both as vector spaces over $\mathbb{Q}$.
\end{solution}



\begin{problem}[4.5]
Prove that the groups $(\mathbb{R}\setminus\{0\},\cdot)$ and $(\mathbb{C}\setminus\{0\}, \cdot)$ are not isomorphic.
\end{problem}

\begin{solution}
	Suppose $f:\mathbb{R}\rightarrow\mathbb{C}$ is an isomorphism. Then there exists a real number $x$ such that $f(x)=i$. 
	\[
	f(x^4)=f(x)^4=i^4=1.
	\]
	Since isomorphism preserves the identity, we have
	\[
	f(1)=1=f(x^4).
	\]
	which indicates $x^4=1$. Noticing that $x\in\mathbb{R}$, there must be $x^2=1$. Now we see
	\[
	f(1)=f(x^2)=f(x)^2=i^2=-1,
	\]
	which derives a contradiction. Thus we can conclude that groups $(\mathbb{R}\setminus\{0\},\cdot)$ and $(\mathbb{C}\setminus\{0\}, \cdot)$ are not isomorphic.

\end{solution}



\begin{problem}[4.6]
	We have seen that $(\mathbb{R}, +)$ and $(\mathbb{R}_{>0},\cdot)$ are isomorphic (Example 4.4). Are the groups $(\mathbb{Q}, +)$ and $(\mathbb{Q}_{>0},\cdot)$ isomorphic?
\end{problem}
\begin{solution}
	Suppose $f:\mathbb{Q}\rightarrow\mathbb{Q}_{>0}$ is an isomorphism. 
	Since isomorphism preserves the multiplication, we have
	\[
	f(1)=f\left(n\cdot\frac{1}{n}\right)=f\left(\frac{1}{n}\right)^n\quad(n\in\mathbb{Z}_{>0}),
	\]
	which implies
	\[
	f\left(\frac{1}{n}\right)=f(1)^{\frac{1}{n}}.
	\] 
	Assume $f(1)=\dfrac{p}{q}=\dfrac{p_1^{r_1}p_2^{r_2}\cdots p_k^{r_k}}{q_1^{s_1}q_2^{s_2}\cdots q_l^{s_l}}$ where $p_i,q_i$ are pairwise distinct positive prime numbers. Then let $M=\max\{p,q\}+1>\max\{r_1,\cdots,r_k,s_1,\cdots,s_l\}$. Thus we assert
	\[
	f\left(\frac{1}{M}\right)=\left(\dfrac{p_1^{r_1}p_2^{r_2}\cdots p_k^{r_k}}{q_1^{s_1}q_2^{s_2}\cdots q_l^{s_l}}\right)^{\frac{1}{M}}\notin\mathbb{Q},
	\]
	which can be proved by contradiction. Suppose 
	\[
	\left(\dfrac{p}{q}\right)^{\tfrac{1}{M}}=\dfrac{a}{b}\in\mathbb{Q}
	\] 
	or say
	\[
	pb^M=qa^M,
	\]
	where $a$, $b$ are coprime. Note $b^M$, $a^M$ are also coprime and the prime factorization of $a^M$ can be written as $a_1^{Mt_1}a_2^{Mt_2}\cdots a_j^{Mt_j}$ where $a_i$ are pairwise distinct positive prime numbers. That forces 
	\[
	p=p_1^{r_1}p_2^{r_2}\cdots p_k^{r_k}=N\cdot a_1^{Mt_1}a_2^{Mt_2}\cdots a_j^{Mt_j}.
	\]
	Noticing that $a_i$ must coincide with one number in $\{p_1,p_2,\cdots p_k\}$, we can assume $a_1=p_1$ without loss of generality. However, since $M>\max\{r_1,\cdots,r_k\}$, we see the exponent of $p_1$ is distinct from that of $a_1$, which violates the unique factorization property of $\mathbb{Z}$. Hence we get a contradiction and conclude $	f\left(\frac{1}{M}\right)\notin \mathbb{Q}$. Moreover, it contradicts our assumption that $f:\mathbb{Q}\rightarrow\mathbb{Q}_{>0}$ is an isomorphism. Eventually we show that the groups $(\mathbb{Q}, +)$ and $(\mathbb{Q}_{>0},\cdot)$ are not isomorphic. 

	

	
\end{solution}


   



\begin{problem}[4.7]	
	Let $G$ be a group. Prove that the function $G\rightarrow G$ defined by $g\mapsto g^{-1}$ is a homomorphism if and only if $G$ is abelian. Prove that $g\mapsto g^2$ is a homomorphism
	if and only if G is abelian.
\end{problem}
\begin{solution}
	Given the function
	\[
	\begin{aligned}
	f\;: G&\longrightarrow G\\  
	g&\longmapsto g^{-1}
	\end{aligned}
	\]
	we have
	\[
	f(g_1g_2)=(g_1g_2)^{-1}=g_2^{-1}g_1^{-1},\quad	f(g_1)f(g_2)=g_1^{-1}g_2^{-1}.
	\]
	If $G$ is abelian, it is clear to see $f(g_1g_2)=f(g_1)f(g_2)$. If $f$ is a homomorphism, $\forall h_1,h_2\in G$,
	\[
	h_1h_2=(h_2^{-1}h_1^{-1})^{-1}=f(h_2^{-1}h_1^{-1})=f(h_2^{-1})f(h_1^{-1})=h_2h_1.
	\]
	Given the function
	\[
	\begin{aligned}
	h\;: G&\longrightarrow G\\  
	g&\longmapsto g^{2}
	\end{aligned}
	\]
	we have
	\[
	h(g_1g_2)=(g_1g_2)^{2}=g_1g_2g_1g_2,\quad h(g_1)h(g_2)=g_1^2g_2^2=g_1g_1g_2g_2.
	\]
	If $G$ is abelian, it is clear to see $h(g_1g_2)=h(g_1)h(g_2)$. If $h$ is a homomorphism, by cancellation we have 
	\[
	h(g_1g_2)=h(g_1)h(g_2)\implies g_2g_1=g_1g_2.
	\]
\end{solution}



\begin{problem}[4.8]
Let $G$ be a group, and $g\in G$. Prove that the function $\gamma_g : G\rightarrow G$ defined by $(\forall a \in G) : \gamma_g(a) = gag^{-1}$ is an automorphism of $G$. (The automorphisms $\gamma_g$ are
called \textquoteleft inner' automorphisms of $G$.) Prove that the function $G\rightarrow \mathrm{Aut}(G)$ defined by $g \mapsto \gamma_g$ is a homomorphism. Prove that this homomorphism is trivial if and only if $G$ is abelian.
\end{problem}
\begin{solution}
Since
\[
\gamma_g(ab)=gabg^{-1}=gag^{-1}gbg^{-1}=\gamma_g(a)\gamma_g(b),
\]
$\gamma_g$ is an automorphism of $G$. 
For all $a\in G$, we have
\[
\gamma_{g_1g_2}(a)=g_1g_2ag_2^{-1}g_1^{-1}=\gamma_{g_1}(g_2ag_2^{-1})=(\gamma_{g_1}\circ \gamma_{g_2})(a),
\]
which implies $\gamma_{g_1g_2}=\gamma_{g_1}\circ \gamma_{g_2}$ and $g \mapsto\gamma_g$ is a homomorphism. If $G$ is abelian, for all $g$ the homomorphism
\[
\gamma_g(a) = gag^{-1}=gg^{-1}a=a
\] 
is the identity in $\mathrm{Aut}(G)$. That is, the homomorphism $g \mapsto \gamma_g$ is  trivial. If the homomorphism $g \mapsto \gamma_g$ is  trivial, we have for all $g,a\in G$,
\[
gag^{-1}=a,
\] 
which implies for all $a,b\in G$,
\[
ab=bab^{-1}b=ba.
\]
Thus we show the homomorphism $g \mapsto \gamma_g$ is trivial if and only if $G$ is abelian.
\end{solution}




\begin{problem}[4.9]
Prove that if $m$, $n$ are positive integers such that $\gcd(m, n) = 1$, then $C_{mn}\cong C_{m}\times C_n$.
\end{problem}
\begin{solution}
Define a function
\[
\begin{aligned}
\varphi\;: C_{m}\times C_n&\longrightarrow C_{mn}\\  
([a]_m,[b]_n)&\longmapsto [anp+bmq]_{mn}
\end{aligned}
\]	
where $[pn]_m=[1]_m$ and $[qm]_n=[1]_n$, as $\gcd(m, n) = 1$ guarantees the existence of $p,q$ (see textbook p56). First of all, we have to check whether $\varphi$ is well-defined. Note that 		
\[
[(anp_1+bmq_1)-(anp_2+bmp_2)]_{m}=[a(p_1n-p_2n)+b(q_1m-q_2m)]_{m}=[0]_m
\]	
\[
[(anp_1+bmq_1)-(anp_2+bmp_2)]_{n}=[a(p_1n-p_2n)+b(q_1m-q_2m)]_{n}=[0]_n
\]
and $\gcd(m, n) = 1$. Thus we have
\[
[(anp_1+bmq_1)-(anp_2+bmp_2)]_{mn}=[0]_{mn},
\]
or
\[
[anp_1+bmq_1]_{mn}=[anp_2+bmp_2]_{mn}.
\]
Then we show $\varphi$ is a homomorphism.
\begin{align*}
\varphi(([a_1]_m,[b_1]_n)+([a_2]_m,[b_2]_n))&=\varphi([a_1+a_2]_m,[b_1+b_2]_n)\\
&=[(a_1+a_2)np+(b_1+b_2)mq]_{mn}\\
&=[a_1np+b_1mq]_{mn}+[a_2np+b_2mq]_{mn}\\
&=\varphi([a_1]_m,[b_1]_n)+\varphi([a_2]_m,[b_2]_n).
\end{align*}
In order to show $\varphi$ is a monomorphism, we can check 
\begin{align*}
&\varphi([a_1]_m,[b_1]_n)=\varphi([a_2]_m,[b_2]_n)\\
\implies&[a_1np+b_1mq]_{mn}=[a_2np+b_2mq]_{mn}\\
\implies&[(a_1-a_2)np+(b_1-b_2)mq]_{mn}=[0]_{mn}\\
\implies&[(a_1-a_2)np+(b_1-b_2)mq]_{m}=[a_1-a_2]_{m}=[0]_m,\\
&[(a_1-a_2)np+(b_1-b_2)mq]_{n}=[b_1-b_2]_{n}=[0]_n\\
\implies&[a_1]_m=[a_2]_m,\ [b_1]_m=[b_2]_m.
\end{align*}
Since  $|C_{m}\times C_n|= |C_{mn}|=mn$, we can conclude $\varphi$ is an isomorphism. Thus we complete proving $C_{mn}\cong C_{m}\times C_n$.
	
\end{solution}

\subsection{\textsection5. Free groups} 
\begin{problem}[5.1]
	Does the category $\mathscr{F}^A$ defined in \textsection5.2 have final objects? If so, what are they?
\end{problem}
\begin{solution}
	Yes, they are functions from $A$ to any trivial group, for example $T=\{t\}$.
	\[\xymatrix{
		G \ar[r]^{\exists!\varphi} & \{t\}\\
		A\ar[ru]_{e}\ar[u]^{j}  
	}\]
	For any object $(j,G)$ in $\mathscr{F}^A$, the trivial homomorphism $\varphi:g\mapsto t$ is the unique homomorphism such that the diagram commutes. That is, $\Hom((j,G),(e,T))=\{\varphi\}$.
\end{solution}


\begin{problem}[5.2]
	Since trivial groups $T$ are initial in $\Grp$, one may be led to think that $(e,T)$ should be initial in $\mathscr{F}^A$, for every $A$: $e$ would be defined by sending every element of $A$ to the (only) element in $T$ ; and for any other group $G$, there is a unique homomorphism $T\rightarrow G$. Explain why $(e, T)$ is not initial in $\mathscr{F}^A$ (unless $A =\emptyset$).
\end{problem}
\begin{solution}
	Let $G=C_2=\{[0]_2,[1]_2\}$. Note that $\varphi\circ e(A)$ must be the trivial subgroup $\{[0]_2\}$. If $x\in A$ and $j(x)=[1]_2$, we see $\varphi\circ e\ne j$ and the following diagram does not commute. 
	\[\xymatrix{
		T\ar[r]^{\varphi} & G\\
		A\ar[ru]_{j}\ar[u]^{e}  
	}\]
	That implies $(e, T)$ is not initial in $\mathscr{F}^A$ unless $A =\emptyset$.
\end{solution}



\begin{problem}[5.3]
	Use the universal property of free groups to prove that the map $j:A\rightarrow F(A)$ is injective, for all sets $A$. (Hint: it suffices to show that for every two elements $a, b$	of $A$ there is a group $G$ and a set-function $f : A\rightarrow G$ such that $f(a)=f(b)$. Why? and how do you construct $f$ and $G$?) [\textsection III.6.3]
\end{problem}
\begin{solution}
	Let $G=S_A$ be the symmetric group over $A$. Define functions $g_a:A\rightarrow A,\;x\mapsto a$ sending every element of $A$ to $a$.
	Since $g_a\in S_A$, we can define an injection
	\[
	\begin{aligned}
	f\;: A&\longrightarrow S_A\\  
	a&\longmapsto g_a
	\end{aligned}
	\]
	In light of the commutative diagram
	\[\xymatrix{
		F(A) \ar[r]^{\varphi} &S_A\\
		A\ar[ru]_{f}\ar[u]^{j}  
	}\]	
	we have $\forall a,b\in A$,
	\[
	j(a)=j(b)\implies\varphi(j(a))=\varphi(j(b))\implies f(a)=f(b)\implies a=b.
	\]
	
	
\end{solution}

\begin{problem}[5.4]
	In the \textquoteleft concrete’ construction of free groups, one can try to reduce words by performing cancellations in any order; the \textquoteleft elementary reductions' used in the text(that is, from left to right) is only one possibility. Prove that the result of iterating cancellations on a word is independent of the order in which the cancellations are performed. Deduce the associativity of the product in $F(A)$ from this. [\textsection 5.3]
\end{problem}
\begin{solution}
	We use induction on the length of $w$. If $w$ is reduced, there is nothing to show. If not, there must be some pair of symbols that can be cancelled, say the underlined pair	
	$$w = \cdots \underline{xx}^{-1}\cdots.$$
	(Let's allow $x$ to denote any element of $A'$, with the understanding that if $x = a^{-1}$ then $x^{-1} = a$.) If we show that we can obtain every reduced form of $w$ by cancelling the pair $xx^{-1}$ first, the proposition will follow by induction, because the word $w^* = \cdots  \underline{\cancel{x}\cancel{x}}^{-1}\cdots$ is shorter.
	
	Let $w_0$ be a reduced form of $w$. It is obtained from $w$ by some sequence of cancellations. The first case is that our pair $xx^{-1}$ is cancelled at some step in this sequence. If so, we may as well cancel $xx^{-1}$ first. So this case is settled. On the other hand, since $w_0$ is reduced, the pair $xx^{-1}$ can not remain in $w_0$. At least one of the two symbols must be cancelled at some time. If the pair itself is not cancelled, the first cancellation involving the pair must look like
	\[
	 \cdots \cancel{x}^{-1}\underline{\cancel{x}x}^{-1}\cdots\quad \text{  or  }\quad  \cdots \underline{x\cancel{x}}^{-1}\cancel{x}\cdots
	\]
	Notice that the word obtained by this cancellation is the same as the one obtained by	cancelling the pair $xx^{-1}$. So at this stage we may cancel the original pair instead. Then we are back in the first case, so the  proposition is proved.

\end{solution}

\begin{problem}[5.5]
Verify explicitly that $H^{\oplus A}$ is a group.
\end{problem}
\begin{solution}
Assume the $A$ is a set and $H$ is an abelian group. $H^{\oplus A}$ are defined as follows 
\[
H^{\oplus A}:=\{\alpha:A\rightarrow H|\alpha(a)\ne e_H \text{ for only finitely many elements }a\in A\} .
\]
Now that $H^{\oplus A}\subset H^A:=\Hom_{\mathsf{Set}}(A,H)$, we can first show $(H^A,+)$ is a group, where for all $\phi,\psi\in H^A$, $\phi+\psi$ is defined by
\[
(\forall a\in A) : (\phi + \psi)(a) := \phi(a) + \psi(a) .
\]
Here is the verification:
\begin{itemize}
	\item Identity: Define a function $\varepsilon:A\rightarrow H,a\mapsto e_H$ sending all elements in $A$ to $e_H$. Then for any $\alpha\in H^A$ we have
	\[
	(\forall a\in A) : (\alpha + \varepsilon)(a) = \alpha(a) + \varepsilon(a)=\alpha(a),
	\]
	which is $\alpha + \varepsilon=\alpha$. Because of the commutativity of the operation $+$ defined on $H^A$, $\varepsilon$ is the identity indeed.
	\item Associativity: This follows by the associativity in $H$:
	\[
	(\forall a\in A) : ((\alpha + \beta)+\gamma)(a) = (\alpha + \beta)(a) + \gamma(a)= \alpha(a)+ (\beta+ \gamma)(a)=(\alpha +(\beta+\gamma))(a).
	\]
	\item Inverse: Every function $\phi\in H^A$ has inverse $-\phi$ defined by 
	\[
	(\forall a\in A) :(-\phi)(a) = -\phi(a).
	\]
\end{itemize}
	Thus $H^A$ makes a group. 
	
	Then it is time to show $H^{\oplus A}$ is a subgroup of $H^A$. For all $\alpha,\beta\in H^{\oplus A}$, let $N_\alpha=\{a\in A|\alpha(a)\ne e_H\}$, $N_\beta=\{a\in A|\beta(a)\ne e_H\}$, $N_{\alpha-\beta}=\{a\in A|(\alpha - \beta)(a)\ne e_H\}$. Since
	\[
	(\forall a\in A) :(\alpha - \beta)(a) = \alpha(a) - \beta(a),
	\]
	we have 
	\[
	(\alpha - \beta)(a)\ne e_H\implies \alpha(a)\ne e_H\text{ or }\beta(a)\ne e_H,
	\]
	which implies $N_{\alpha-\beta}\subset N_\alpha\cup N_\beta$. Note that $N_\alpha$, $N_\beta$ are both finite sets, which forces $N_{\alpha-\beta}$ to be finite. So there must be $\alpha - \beta\in H^{\oplus A}$. Now we see $H^{\oplus A}$ is closed under additions and inverses. And $e_{H^A}=\varepsilon\in H^{\oplus A}$ means that $H^{\oplus A}$ is nonempty. Finally we can conclude $H^{\oplus A}$ is a subgroup of $H^A$.
\end{solution}

\begin{problem}[5.6]
Prove that the group $F(\{x, y\})$ (visualized in Example 5.3) is a coproduct $\mathbb{Z}*\mathbb{Z}$ of $\mathbb{Z}$ by itself in the category $\Grp$. (Hint: with due care, the universal property for one turns into the universal property for the other.) [\textsection 3.4, 3.7, 5.7]
\end{problem}
\begin{solution}
Define two homomorphisms 
\begin{align*}
i_1:\mathbb{Z}\longrightarrow F(\{x, y\}),\quad n\longmapsto x^n,\\ i_2:\mathbb{Z}\longrightarrow F(\{x, y\}),\quad n\longmapsto y^n.
\end{align*}	
We need to show that for any group $G$ with two homomorphisms $f_1,f_2:\mathbb{Z}\rightarrow G$, there exists a unique homomorphism $\varphi$ such that the following diagram commutes.
\[\xymatrix{
	\mathbb{Z}\ar[rd]^{f_1}\ar[d]_{i_1}  \\
	F(\{x, y\})\ar[r]^{\quad\varphi} &  G\\
	\mathbb{Z}\ar[ru]_{f_2}\ar[u]^{i_2}  
}\]
Given the notation of indicator function
\[\mathbf{1}_A(x) :=
\begin{cases}
	1 &\text{if } x \in A, \\
	0 &\text{if } x \notin A,
\end{cases}
\]
we can define a function
\begin{align*}
\varphi:\ &F(\{x, y\})\longrightarrow G,\\
&z_1^{n_1}\cdots z_k^{n_k} \longmapsto f_1(n_1)^{\mathbf{1}_{\{x\}}(z_1)}f_2(n_1)^{\mathbf{1}_{\{y\}}(z_1)}\cdots f_1(n_k)^{\mathbf{1}_{\{x\}}(z_n)}f_2(n_k)^{\mathbf{1}_{\{y\}}(z_n)},\ z_i\in\{x,y\}
\end{align*}
and check that it is a homomorphism indeed. For all $n\in \mathbb{Z}$, we have
\begin{align*}
(\varphi\circ i_1)(n)=\varphi(x^n)=f_1(n),\\
(\varphi\circ i_2)(n)=\varphi(y^n)=f_2(n),
\end{align*}	
that is, the diagram commutes. Now we see $\varphi$ exists. For the uniqueness of $\varphi$, let $\varphi^*$ be another homomorphism that makes diagram commute. For all $z_1^{n_1}\cdots z_k^{n_k}\in F(\{x, y\}),\,z_i\in\{x, y\}$, we have
\begin{align*}
\varphi^*(z_1^{n_1}\cdots z_k^{n_k})&=\varphi^*(z^{n_1})\cdots \varphi^*(z^{n_k})\\
&=\varphi^*(i_1(n_1))^{\mathbf{1}_{\{x\}}(z_1)}\varphi^*(i_2(n_1))^{\mathbf{1}_{\{y\}}(z_1)}\cdots \varphi^*(i_1(n_k))^{\mathbf{1}_{\{x\}}(z_1)}\varphi^*(i_2(n_k))^{\mathbf{1}_{\{y\}}(z_1)}\\
&=f_1(n_1)^{\mathbf{1}_{\{x\}}(z_1)}f_2(n_1)^{\mathbf{1}_{\{y\}}(z_1)}\cdots f_1(n_k)^{\mathbf{1}_{\{x\}}(z_n)}f_2(n_k)^{\mathbf{1}_{\{y\}}(z_n)}\\
&=\varphi(z_1^{n_1}\cdots z_k^{n_k}).
\end{align*}	
To sum up, we have shown that the group $F(\{x, y\})$ is a coproduct $\mathbb{Z}*\mathbb{Z}$ of $\mathbb{Z}$ by itself in the category $\Grp$.
\end{solution}	
	
\begin{problem}[5.7]
Extend the result of Exercise 5.6 to free groups $F(\{x_1,\dots, x_n\})$ and to free
abelian groups $F^{ab}(\{x_1,\dots, x_n\})$. [§3.4, §5.4]	
\end{problem}
\begin{solution}

Let $*$ be coproduct. Then we have $\underbrace{\mathbb{Z}*\mathbb{Z}*\dots*\mathbb{Z}}_{n\text{ times}}\cong F(\{x_1,\dots, x_n\})$, as the following diagram demonstrates:
\[\xymatrix{
		\mathbb{Z}\ar[rrrd]^{f_1}\ar[rd]_{\hspace{-5ex}{i_1}} \\
		\hspace{4.2em}
		\raisebox{2.35ex}{\vdots}
		\raisebox{0ex}{\hspace{-0.63ex}\vdots}
		\raisebox{-2.35ex}{\hspace{-0.63ex}\vdots}
		\hspace{4em}
		\vdots 
		& F(\{x_1,\dots, x_n\})\ar[rr]^{\varphi} 
		& &G\\
		\mathbb{Z}\ar[rrru]_{f_n}\ar[ru]^{\hspace{-5ex}i_n}  
}\]
~\\
Dually, let $\times$ be product. Then we have $\underbrace{\mathbb{Z}\times\mathbb{Z}\times\cdots\times\mathbb{Z}}_{n\text{ times}}\cong F^{ab}(\{x_1,\cdots, x_n\})$, as the following diagram demonstrates:	
\[\xymatrix@R=12.7pt{
	 	 & & &&\mathbb{Z} \\
	 	 & & &&\\
		G\ar[rrrruu]^{f_1}\ar[rrrrdd]_{f_n}\ar[rr]^{\hspace{-1.5em}\varphi} & &F^{ab}(\{x_1,\dots,x_n\})\ar[rruu]_{\hspace{1ex}\pi_1}\ar[rrdd]^{\hspace{1ex}\pi_n}&
		&\hspace{-4.04em}
		\vdots
		\hspace{3.7em}
		\raisebox{2.35ex}{\vdots}
		\raisebox{0ex}{\hspace{-0.63ex}\vdots}
		\raisebox{-2.35ex}{\hspace{-0.63ex}\vdots}\\
		 & & &&\\
		 & & &&\mathbb{Z} 
}\]

\end{solution}


\begin{problem}[5.8]
Still more generally, prove that $F(A\amalg B)=F(A)*F(B)$ and that $F^{ab}(A\amalg B) =F^{ab}(A)\oplus F^{ab}(B)$ for all sets $A, B$. (That is, the constructions $F$, $F^{ab}$ 'preserve coproducts'.)
\end{problem}
\begin{solution}
In order to show $F(A)*F(B)$ is a free group generated by $A\amalg B$, we should first set an appropriate function $\psi:A\amalg B\rightarrow F(A)*F(B)$ and then prove that given any $(\theta,G)$ there exists a unique group homomorphism $g$ such that the following diagram commutes.
	\[\xymatrix{
		A\amalg B\ar[rr]^{\hspace{-1em}\psi}\ar@/_2pc/[rrrr]_{\theta} && F(A)*F(B)\ar@{-->}[rr]^{\hspace{1em}\exists!g}&&G\\
	}\]
The complete proof can be divided into three steps, by dividing the following diagram into parts.\\
\[\xymatrix{
	A\ar[r]^{j_1}\ar[d]^{i_1}&F(A)\ar@{-->}[rd]^{\varphi_1}\ar[d]^{f_1}  \\
	A\amalg B\ar@{-->}[r]^{\hspace{-1em}\psi}\ar@/_2.4pc/[rr]^{\hspace{-2em}\theta}&F(A)*F(B)\ar@{-->}[r]^{\hspace{1.3em}g}&G\\	B\ar[r]^{j_1}\ar[u]_{i_2}&F(B)\ar@{-->}[ru]_{\varphi_2}\ar[u]_{f_2}    \\ 
}\]
\noindent\textbf{Step 1. Construct $\psi:A\amalg B\longrightarrow F(A)*F(B)$.} 

\noindent Define injective functions 
\begin{align*}
i_1:\ &A\longrightarrow A\amalg B,\quad a\longmapsto (a,1),\\ 
i_2:\ &B\longrightarrow A\amalg B,\quad b\longmapsto (b,2),\\
j_1:\ &A\longrightarrow F(A),\quad a\longmapsto a,\\ 
j_2:\ &B\longrightarrow F(B),\quad b\longmapsto b.
\end{align*}
Let $f_1,f_2$ be the homomorphisms specified by the coproduct in $\Grp$. Since $A\amalg B$ is a coproduct in $\Set$, the universal property guarantees a unique mapping $\psi:A\amalg B\rightarrow F(A)*F(B)$ such that the following diagram commutes
\[\xymatrix{
	A\ar[rr]^{j_1}\ar[d]^{i_1}&&F(A)\ar[d]^{f_1}  \\
	A\amalg B\ar@{-->}[rr]^{\exists!\psi}  &&F(A)*F(B)\\
	B\ar[rr]^{j_1}\ar[u]_{i_2}&&F(B)\ar[u]_{f_2}  \\ 
}\]
That is,
\[
\exists!\ \psi:A\amalg B\longrightarrow F(A)*F(B)\quad(\psi\circ i_1=f_1\circ j_1)\wedge (\psi\circ i_2=f_2\circ j_2).
\]
\noindent\textbf{Step 2. Prove the existence of $g$.}
\[\xymatrix{
	A\ar[r]^{j_1}\ar[d]^{i_1}&F(A)\ar@{-->}[rd]^{\exists!\varphi_1}  \\
	A\amalg B\ar[rr]^{\theta}&  &G\\
	B\ar[r]^{j_1}\ar[u]_{i_2}&F(B)\ar@{-->}[ru]_{\exists!\varphi_2}    \\ 
}\]
Given some $(\theta,G)$, according to the universal property of free groups $F(A)$, $F(B)$, we have
\begin{align*}
\exists!\ \varphi_1:F(A)\longrightarrow G\quad(\varphi_1\circ j_1=\theta\circ i_1),\\ 
\exists!\ \varphi_2:F(B)\longrightarrow G\quad(\varphi_2\circ j_2=\theta\circ i_2).
\end{align*} 
\[\xymatrix{
	F(A)\ar[d]^{f_1}\ar@{-->}[rd]^{\varphi_1}  \\
    F(A)*F(B)\ar@{-->}[r]^{\hspace{1.3em}\exists!g} &G\\
	F(B)\ar[u]_{f_2}\ar@{-->}[ru]_{\varphi_2}    \\ 
}\]
Then according to the universal property of coproduct $F(A)*F(B)$ in $\Grp$, we have
\[
\exists!\ g:F(A)*F(B)\longrightarrow G\quad(g\circ f_1= \varphi_1)\wedge (g\circ f_2= \varphi_2).
\]
The commutative diagram tells us
\begin{align*}
g\circ \psi\circ i_1=g\circ f_1\circ j_1=\varphi_1\circ j_1=\theta\circ i_1,\\
g\circ \psi\circ i_2=g\circ f_2\circ j_2=\varphi_2\circ j_2=\theta\circ i_2.
\end{align*}
Note that $A\amalg B=i_1(A)\cup i_2(B)$. For all $x\in A\amalg B$, $x$ must be either $i_1(a)$ or $i_2(b)$. If $x=i_1(a)$, then
\[
g\circ \psi(x)=g\circ \psi\circ i_1(a)=\theta\circ i_1(a)=\theta(x).
\]  
If $x=i_2(b)$, then
\[
g\circ \psi(x)=g\circ \psi\circ i_2(b)=\theta\circ i_2(b)=\theta(x).
\]  
Hence we show that given some $(\theta,G)$ there exists $g:F(A)*F(B)\longrightarrow G$ such that $g\circ \psi=\theta$.\\

\noindent\textbf{Step 3. Prove the uniqueness of $g$.}

\noindent Assume there exists another homomorphism $h$ such that $h\circ \psi=\theta$. We have
\begin{align*}
h\circ f_1\circ j_1=h\circ \psi\circ i_1=\theta\circ i_1,\\
h\circ f_2\circ j_2=h\circ \psi\circ i_2=\theta\circ i_2.
\end{align*} 
Since
\begin{align*}
\exists!\ \varphi_1:F(A)\longrightarrow G\quad(\varphi_1\circ j_1=\theta\circ i_1),\\ 
\exists!\ \varphi_2:F(B)\longrightarrow G\quad(\varphi_2\circ j_2=\theta\circ i_2),
\end{align*} 
there must be
\begin{align*}
h\circ f_1&=\varphi_1,\\
h\circ f_2&=\varphi_2.
\end{align*}
Again by universal property 
\[
\exists!\ g:F(A)*F(B)\longrightarrow G\quad(g\circ f_1= \varphi_1)\wedge (g\circ f_2= \varphi_2)
\]
we get $h=g$, which implies $g$ is unique.\\

\noindent\textbf{Conclusion.}

\noindent To sum up, we prove that there exists a unique group homomorphism $g$ such that the first diagram in this proof commutes. As a result, we have $F(A\amalg B)=F(A)*F(B)$. Note that if $\Grp$ turns into $\Ab$, the method of diagram chasing applied here also works. In the light of the following diagram, we can get $F^{ab}(A\amalg B) =F^{ab}(A)\oplus F^{ab}(B)$ step by step. 
\[\xymatrix{
	A\ar[r]^{j_1}\ar[d]^{i_1}&F^{ab}(A)\ar@{-->}[rd]^{\varphi_1}\ar[d]^{f_1}  \\
	A\amalg B\ar@{-->}[r]^{\hspace{-2em}\psi}\ar@/_2.4pc/[rr]^{\hspace{-2em}\theta}&F^{ab}(A)\oplus F^{ab}(B)\ar@{-->}[r]^{\hspace{2em}g}&G\\	B\ar[r]^{j_1}\ar[u]_{i_2}&F^{ab}(B)\ar@{-->}[ru]_{\varphi_2}\ar[u]_{f_2}    \\ 
}\]
\end{solution}

\begin{problem}[5.9]
 Let $G = \mathbb{Z}^{\oplus\mathbb{N}}$. Prove that $G\times G\cong G$.
\end{problem}
\begin{solution}
Define a function
\begin{align*}
\varphi:G\times G&\longrightarrow G\\
((a_1,a_2,\cdots),(b_1,b_2,\cdots))&\longmapsto (a_1,b_1,a_2,b_2,\cdots)
\end{align*}
It plain to check that $\varphi$ is a homomorphism
\begin{align*}
&\varphi[((a_1,a_2,\cdots),(b_1,b_2,\cdots))+((a_1',a_2',\cdots),(b_1',b_2',\cdots))]\\
=&\varphi[((a_1+a_1',a_2+a_2',\cdots),(b_1+b_1',b_2+b_2',\cdots))]\\
=&(a_1+a_1',b_1+b_1',a_2+a_2',b_2+b_2',\cdots)\\
=&(a_1,b_1,a_2,b_2,\cdots)+(a_1',b_1',a_2',b_2',\cdots)\\
=&\varphi[((a_1,a_2,\cdots),(b_1,b_2,\cdots))]+\varphi[((a_1',a_2',\cdots),(b_1',b_2',\cdots))].
\end{align*}
Since $\ker\varphi=\{(0,0,\cdots)\}$ and $|G\times G|=|G|=\aleph_0$, we can conclude that $\varphi$ is an isomorphism and accordingly $G\times G\cong G$.
\end{solution}

\subsection{\textsection6. Subgroups} 

\begin{problem}[6.1]
$\neg$ (If you know about matrices.) The group of invertible $n \times n$ matrices with entries in R is denoted $\GL_n(\R)$ (Example 1.5). Similarly, $\GL_n(\C)$ denotes the group of $n \times n$ invertible matrices with complex entries. Consider the following sets of
matrices:
\begin{itemize}
	\item $\SL_n(\R) = \{M \in \GL_n(\R) | \det(M) = 1\}$;
	\item $ \SL_n(\C) = \{M \in \GL_n(\C) | \det(M) = 1\}$;
	\item $\mathrm{O}_n(\R) = \{M \in \GL_n(\R) |MM^t = M^tM = I_n\}$;
	\item $\mathrm{SO}_n(\R) = \{M \in \mathrm{O}_n(\R) | \det (M) = 1\}$;
	\item  $\mathrm{U}_n(\C) = \{M \in \GL_n(\C) |MM^\dag = M^\dag M = I_n\}$;
	\item  $\mathrm{SU}_n(\C) = \{M \in \mathrm{U}_n(\C) | \det(M) = 1\}$.
\end{itemize}

Here In stands for the $n \times n$ identity matrix, $M^t$ is the transpose of $M$, $M\dag$ is the
conjugate transpose of $M$, and $\det(M)$ denotes the determinant of $M$. Find all possible inclusions among these sets, and prove that in every case the smaller set is a subgroup of the larger one.

These sets of matrices have compelling geometric interpretations: for example, $\mathrm{SO}^3(\R)$ is the group of ‘rotations’ in $\R^3$. [8.8, 9.1, III.1.4, VI.6.16]
\end{problem}
\begin{solution}
The following diagram commutes, where all arrows are inclusions.
\[\xymatrix{
	\GL_n(\R)\ar[r] & \GL_n(\C)\\
	\SL_n(\R)\ar[r]\ar[u] & \SL_n(\C)\ar[u]\\
	\mathrm{O}_n(\R)\ar[r]\ar[u] & \mathrm{U}_n(\C)\ar[u]\\
	\mathrm{SO}_n(\R)\ar[r]\ar[u] & \mathrm{SU}_n(\C) \ar[u]
}\]
\end{solution}

\begin{problem}[6.2]
$\neg$ Prove that the set of $2\times2$ matrices
\[
\begin{pmatrix}
	a& b\\
	0& d\\
\end{pmatrix}
\]
with $a$, $b$, $d$ in $\C$ and $ad\ne 0$ is a subgroup of $\GL_2(\C)$. More generally, prove that the set of $n\times n$ complex matrices $(a_{ij})_{1\le i,j\le n}$ with $a_{ij} = 0$ for $i>j$, and $a_{11}\cdots a_{nn}\ne 0$, is a subgroup of $\GL_n(\C)$. (These matrices are called 'upper triangular', for evident reasons.) [IV.1.20]
\end{problem}
\begin{solution}
Let $A, B$ are $n\times n$ upper triangular matrices. If $i>j$,
\[
(AB)_{ij}=\sum_{k=1}^{n}a_{ik}b_{kj}=\sum_{k=1}^{i-1}a_{ik}b_{kj}+\sum_{k=i}^{n}a_{ik}b_{kj}=\sum_{k=1}^{i-1}0b_{kj}+\sum_{k=i}^{n}a_{ik}0=0,
\]
which means the set of upper triangular matrices is closed with respect to the matrix multiplication. Thus it is a subgroup of $\GL_n(\C)$.
\end{solution}


\begin{problem}[6.3]
$\neg$ Prove that every matrix in $\mathrm{SU}_2(\C)$ may be written in the form
\[
\begin{pmatrix}
a + bi& c + di\\
-c + di& a-bi\\
\end{pmatrix}
\]
where $a, b, c, d \in \R$ and $a^2 + b^2 + c^2 + d^2 = 1$. (Thus, $\mathrm{SU}_2(\C)$ may be realized as a three-dimensional sphere embedded in $\R^4$; in particular, it is simply connected.)[8.9, III.2.5]
\end{problem}
\begin{solution}
Let
	\[
	A=
	\begin{pmatrix}
	a_{11}& a_{12}\\
    a_{21}& a_{22}\\
	\end{pmatrix}
	\in\mathrm{SU}_2(\C)
	\]
and we have
\[
AA^\dag=
\begin{pmatrix}
a_{11}& a_{12}\\
a_{21}& a_{22}\\
\end{pmatrix}
\begin{pmatrix}
\overline{a_{11}}& \overline{a_{21}}\\
\overline{a_{12}}& \overline{a_{22}}\\
\end{pmatrix}
=
\begin{pmatrix}
|a_{11}|^2+|a_{12}|^2& a_{11}\overline{a_{21}}+a_{12}\overline{a_{22}}\\
 a_{21}\overline{a_{11}}+a_{22}\overline{a_{12}}& |a_{21}|^2+|a_{22}|^2\\
\end{pmatrix}
=
\begin{pmatrix}
1& 0\\
0& 1\\
\end{pmatrix}
\]
and
\[
\det(A)=
\begin{vmatrix}
a_{11}& a_{12}\\
a_{21}& a_{22}\\
\end{vmatrix}=a_{11}a_{22}-a_{12}a_{21}=1
\]
\[
\overline{a_{11}}\overline{a_{12}}=\overline{a_{11}}\overline{a_{12}}
\begin{vmatrix}
a_{11}& a_{12}\\
a_{21}& a_{22}\\
\end{vmatrix}
=
\begin{vmatrix}
|a_{11}|^2& |a_{12}|^2\\
a_{21}\overline{a_{11}}& a_{22}\overline{a_{12}}\\
\end{vmatrix}
=
\begin{vmatrix}
|a_{11}|^2&|a_{11}|^2+ |a_{12}|^2\\
a_{21}\overline{a_{11}}&a_{21}\overline{a_{11}}+ a_{22}\overline{a_{12}}\\
\end{vmatrix}
=
\begin{vmatrix}
|a_{11}|^2&1\\
a_{21}\overline{a_{11}}&0\\
\end{vmatrix}
=-a_{21}\overline{a_{11}}
\]
\[
\implies\overline{a_{11}}(\overline{a_{12}}+a_{21})=0
\]
\[
\overline{a_{21}}\overline{a_{22}}=\overline{a_{21}}\overline{a_{22}}
\begin{vmatrix}
a_{11}& a_{12}\\
a_{21}& a_{22}\\
\end{vmatrix}
=
\begin{vmatrix}
a_{11}\overline{a_{21}}& a_{12}\overline{a_{22}}\\
|a_{21}|^2& |a_{22}|^2\\
\end{vmatrix}
=
\begin{vmatrix}
a_{11}\overline{a_{21}}&a_{11}\overline{a_{21}}+ a_{12}\overline{a_{22}}\\
|a_{21}|^2&|a_{21}|^2+ |a_{22}|^2\\
\end{vmatrix}
=
\begin{vmatrix}
a_{11}\overline{a_{21}}&0\\
|a_{21}|^2&1\\
\end{vmatrix}
=a_{11}\overline{a_{21}}
\]
\[
\implies\overline{a_{21}}(\overline{a_{11}}-a_{22})=0
\]
If $\overline{a_{11}}\ne0$, it must be $\overline{a_{12}}+a_{21}=0$. If $\overline{a_{11}}=0$, then $|a_{12}|^2=1$, $a_{12}\overline{a_{22}}=0$ and accordingly $a_{22}=0$. Since $-a_{12}a_{21}=1=a_{12}\overline{a_{12}}$, we also have $\overline{a_{12}}+a_{21}=0$, that is $a_{12}=c+di, a_{21}=-c+di$. Likewise, we can show $\overline{a_{11}}-a_{22}=0$ and $a_{11}=a+bi, a_{22}=a-bi$. And we have
\[
|a_{11}|^2+|a_{12}|^2=a^2 + b^2 + c^2 + d^2 = 1.
\]
\end{solution}


\begin{problem}[6.4]
Let $G$ be a group, and $g\in G$. Verify that the image of the exponential map
$\epsilon_g:\Z \rightarrow G$ is a cyclic group (in the sense of Definition 4.7).
\end{problem}
\begin{solution}
If $|g|=\infty$, then $g^i\ne g^j(i\ne j)$. Define
\[
\varphi:\Z\longrightarrow \epsilon_g(\Z),n\longmapsto g^n
\] 
and we can check it is an isomorphism.

\noindent If $|g|=k$, then $e_G,g,g^2,\cdots,g^{k-1}$ are distinct. Define
\[
\varphi:\Z/k\Z\longrightarrow \epsilon_g(\Z),[n]_k\longmapsto g^n
\]
and we can check it is an isomorphism. 

\noindent Since $\epsilon_g(\Z)$ is isomorphic to $\Z$ or $\Z/k\Z$, we show $\epsilon_g(\Z)$ is a cyclic group.
\end{solution}

\begin{problem}[6.6]
Prove that the union of a family of subgroups of a group $G$ is not necessarily
a subgroup of $G$. In fact:
\begin{itemize}
	\item Let $H$, $H'$ be subgroups of a group $G$. Prove that $H\cup H'$ is a subgroup of $G$ only if $H\subseteq H'$ or $H'\subseteq H$.
	\item On the other hand, let $H_0 \subseteq H_1 \subseteq H_2 \subseteq \cdots$ be subgroups of a group $G$. Prove that $\cup_{i\ge0}H_i$ is a subgroup of $G$.
\end{itemize}

\end{problem}
\begin{solution}
\begin{itemize}
	\item Let $H\cup H'$ be a subgroup of $G$. Suppose neither $H\subseteq H'$ nor $H'\subseteq H$ hold. Let $a\in H-H'$, $b\in H'-H$, $h=ab^{-1}\in H\cup H'$. In the case of $h\in H$, we have $b=h^{-1}a\in H$, contradiction! In the case of $h\in H'$, we have $a=hb\in H'$, contradiction again! Therefore, there must be $H\subseteq H'$ or $H'\subseteq H$.
	\item For all $a,b\in \cup_{i\ge0}H_i$, we can suppose $a\in H_j,b\in H_k$ and we have $a,b\in H_{\max\{j,k\}}$. Then $ab\in H_{\max\{j,k\}}\subseteq \cup_{i\ge0}H_i$, implies that $\cup_{i\ge0}H_i$ is closed and that $\cup_{i\ge0}H_i$ is a subgroup of $G$. 
\end{itemize}	
	
	
\end{solution}



\begin{problem}[6.7]
$\neg$ Show that inner automorphisms (cf. Exercise 4.8) form a subgroup of
$\Aut(G)$; this subgroup is denoted $\Inn(G)$. Prove that $\Inn(G)$ is cyclic if and only if $\Inn(G)$ is trivial if and only if $G$ is abelian. (Hint: Assume that $\Inn(G)$ is cyclic; with notation as in Exercise 4.8, this means that there exists an element $a \in G$ such that $\forall g \in  G\ \exists n\ \in Z\ \gamma_g = \gamma^n_a$. In particular, $gag^{-1} = a^naa^{-n} = a$. Thus a commutes with every $g$ in $G$. Therefore...) Deduce that if $\Aut(G)$ is cyclic then $G$ is abelian. [7.10, IV.1.5]
\end{problem}
\begin{solution}
With notation as in Exercise 4.8, we assume $\gamma_g\in \Inn(G)$ is defined by
\[
\forall h\in G\;\;(\gamma_g(h)=ghg^{-1}).
\]
\begin{align*}
	&\Inn(G) \text{ is cyclic} \\
\iff&\exists \gamma_a\in \Inn(G),\ \Inn(G)=\langle\gamma_a\rangle\\
\iff&\exists a\in G\; \forall g\in G\; \exists n\in \Z\;(\gamma_g=\gamma_a^n)\\
\implies&\exists a\in G\; \forall g\in G\; \exists n\in \Z\; (\gamma_g(a)=gag^{-1}=\gamma_a^n(a)=a^naa^{-n} = a)\\
\implies&\exists a\in G\; \forall g\in G\; (ga=ag)\\
\implies&\forall h\in G,\gamma_a(h)=aha^{-1}=haa^{-1}=h\\
\implies&\Inn(G)=\langle \mathrm{id}\rangle\\
\implies&\Inn(G) \text{ is trivial} \\
\\
&\Inn(G) \text{ is trivial} \\
\implies&\forall g\in G\;\forall h\in G\,(\gamma_g(h)=ghg^{-1}=h)\\
\implies&\forall g\in G\;\forall h\in G\,(gh=hg)\\
\iff& G \text{ is abelian}\\
\\
&G \text{ is abelian} \\
\implies&\forall g\in G\;\forall h\in G\,(\gamma_g(h)=ghg^{-1}=h)\\
\implies&\Inn(G)=\{\mathrm{id}\}\\
\implies&\Inn(G) \text{ is cyclic}
\end{align*}

\noindent If $\Aut(G)$ is cyclic, its subgroup $\Inn(G)$ is also cyclic. As we have shown, that means $G$ is abelian. 
\end{solution}


\begin{problem}[6.8]
	Prove that an abelian group $G$ is finitely generated if and only if there is a surjective homomorphism
	\[
	\underbrace{\Z\oplus\cdots\oplus\Z}_{n \text{ times}}\twoheadrightarrow G
	\]	
	for some $n$.
\end{problem}
\begin{solution}
	Given any set $H\subseteq G$, there exists a unique homomorphism $\varphi_H$ such that the following diagram commutes. 	
	\[\xymatrix{
		F^{ab}(H)\ar[r]^{\quad\exists!\varphi} & G\\
		H\ar@{^{(}->}[ru]_{i}\ar@{|->}[u]^{j}  
	}\]
	The  homomorphism image $\varphi_H(F^{ab}(H))\le G$ is called the subgroup generated by $H$ in $G$, denoted by $\langle H\rangle$. 
	
	If $G$ is finitely generated, there is a finite subset $G_n\subseteq G$ with $n$ elements such that $\varphi_H(F^{ab}(G_n))=\varphi_H(\Z^{\oplus n})=G$. And $\varphi_H$ is exactly the surjective homomorphism that we need. 
	
	If there is a surjective homomorphism $\psi:\Z^{\oplus n}\twoheadrightarrow G$ for some $n$. Suppose
	\[
	\begin{matrix}
		\psi:\mathbf{1}_i=(0,\cdots,0,&1&,0,\cdots,0)\longmapsto g_i\\
		            &i\text{-th place}&
	\end{matrix}
	\]
	and $G_n=\{g_1,g_2,\cdots,g_n\}$. Then define
	\[
	j:G_n\longrightarrow \Z^{\oplus n},\quad g_i\longmapsto\mathbf{1}_i.
	\]
	We can check the following diagram commutes
	\[\xymatrix{
		\Z^{\oplus n}\ar@{->>}[r]^{\psi} & G\\
		G_n\ar@{^{(}->}[ru]_{i}\ar@{|->}[u]^{j}  
	}\]
	which means $\langle G_n\rangle=\psi(\Z^{\oplus n})$. Since $\psi$ is surjective, we have $\langle G_n\rangle=G$. Hence we show $G$ is finitely generated.
\end{solution}


\begin{problem}[6.9]
Prove that every finitely generated subgroup of $\Q$ is cyclic. Prove that $\Q$ is not finitely generated.
\end{problem}
\begin{solution}	
Given any two rationals  
\begin{align*}
a_1=\frac{p_1}{q_1}\in \Q,(p_1,q_1)=1,  \\
a_2=\frac{p_2}{q_2}\in \Q,(p_2,q_2)=1 ,  
\end{align*}	
there exists $r=\frac{1}{q_1q_2}\in \Q$ such that $\langle a_1,a_2\rangle\le\langle r_1\rangle$. Then for some $a_3$ we have
$\langle a_1,a_2,a_3\rangle\le\langle r_1,a_3\rangle\le\langle r_2\rangle$. In general, let's set $B_n=\{a_1,a_2,\cdots,a_n\}$. If $\langle B_n\rangle\le\langle r_{n-1}\rangle$. we have $\langle B_{n+1}\rangle=\langle B_n,a_{n+1}\rangle\le\langle r_{n-1}a_{n+1}\rangle\le\langle r_n\rangle$. By induction we can prove $\langle a_1,a_2,\cdots,a_n\rangle\le\langle r_{n-1}\rangle
$ for $n\in\N_+$.	
	
	
\end{solution}	
	
\begin{problem}[6.10]	
$\neg$  The set of $2\times2$ matrices with integer entries and determinant 1 is denoted
$\SL_2(\Z)$:
\[
\SL_2(\Z) =
\left\{
\begin{pmatrix}
a & b\\
c & d\\
\end{pmatrix}
\text{ such that } a, b, c, d \in \Z,\;ad-bc = 1
\right\}.
\]
Prove that $\SL_2(\Z)$ is generated by the matrices:
\[
s =
\begin{pmatrix}
0 & -1\\
1 & 0\\
\end{pmatrix}
\text{    and    } 
t =
\begin{pmatrix}
1 & 1\\
0 & 1\\
\end{pmatrix}.
\]
	
\end{problem}
\begin{solution}	
	Let $H$ be the subgroup generated by $s$ and $t$. Given a matrix $m =
	\begin{pmatrix}
		a & b\\
		c & d\\
	\end{pmatrix}$
	in $\SL_2(\Z)$, it suffices to show that we can obtain the identity
	by multiplying $m$ by suitably chosen elements of $H$. Prove that
	\[
	P =
	\begin{pmatrix}
	1 & -p\\
	0 & 1\\
	\end{pmatrix}
	\text{    and    } 
	Q =
	\begin{pmatrix}
	1 & 0\\
	-q & 1\\
	\end{pmatrix}.
	\]
	are in H, and note that
	\[
\begin{pmatrix}
a & b\\
c & d\\
\end{pmatrix}
\begin{pmatrix}
1 & -p\\
0 & 1\\
\end{pmatrix}
=
\begin{pmatrix}
a & b-pa\\
c & d-pc\\
\end{pmatrix}
\text{    and    } 
\begin{pmatrix}
a & b\\
c & d\\
\end{pmatrix}
\begin{pmatrix}
1 & 0\\
-q & 1\\
\end{pmatrix}
=
\begin{pmatrix}
a-qb & b\\
c-qd & d\\
\end{pmatrix}.
\]

	Note that if c and d are both nonzero, one of these two operations may be used
	to decrease the absolute value of one of them. Argue that suitable applications of
	these operations reduce to the case in which c = 0 or d = 0. Prove directly that
	m ∈ H in that case.
\end{solution}		
	
\bibliographystyle{plain}
\bibliography{mybibtex}


\end{document}
