% Compile with PdfLaTeX

\documentclass[12pt,letterpaper,boxed]{hmcpset}

% set 1-inch margins in the document
\usepackage[margin=1in]{geometry}

% include this if you want to import graphics files with /includegraphics

\usepackage{graphicx}
\usepackage{amssymb}
\usepackage[all]{xy}

\usepackage[usenames,dvipsnames]{color}
\usepackage[colorlinks,linkcolor=NavyBlue,anchorcolor=red,citecolor=green]{hyperref}



\usepackage{setspace}

\renewcommand{\baselinestretch}{1.05} 

\renewcommand\appendix{\setcounter{secnumdepth}{-2}}

\newcommand{\Obj}{\mathrm{Obj}} 
\newcommand{\Hom}{\mathrm{Hom}} 
\newcommand{\Grp}{\mathsf{Grp}} 


% info for header block in upper right hand corner
\name{Huyi Chen}
\updatedate{2018/03/31}

\begin{document}
		
\problemlist{\textbf{Algebra, Chapter 0}\\By Paolo Aluffi}

\tableofcontents
\appendix

\section{Chapter I.\quad Preliminaries: Set theory and categories}

\subsection{\textsection3. Categories}
\begin{problem}[3.1]
	Let $\mathsf{C}$ be a category. Consider a structure $\mathsf{C}^{op}$ with:
	\begin{itemize}
	\item $\Obj(\mathsf{C}^{op}) := \Obj(\mathsf{C})$;
	\item for $A$, $B$ objects of $\mathsf{C}^{op}$ (hence, objects of $\mathsf{C}$), $\Hom_{\mathsf{C}^{op}} (A,B) := \Hom_\mathsf{C}(B,A)$
	\end{itemize}
	Show how to make this into a category (that is, define composition of morphisms
	in $\mathsf{C}^{op}$ and verify the properties listed in \textsection3.1).
	Intuitively, the 'opposite' category $\mathsf{C}^{op}$ is simply obtained by 'reversing all the
	arrows' in C. [5.1, \textsection VIII.1.1, \textsection IX.1.2, IX.1.10]
\end{problem}
\begin{solution}
	\begin{itemize}
		\item For every object $A$ of $\mathsf{C}$, there exists one identity morphism $1_A\in\Hom_\mathsf{C}(A,A)$. Since $\Obj(\mathsf{C}^{op}) := \Obj(\mathsf{C})$ and $\Hom_{\mathsf{C}^{op}} (A,A) := \Hom_\mathsf{C}(A,A)$, for every object $A$ of $\mathsf{C}^{op}$, the identity on $A$ coincides with $1_A\in\mathsf{C}$. 
		\item For $A$, $B$, $C$ objects of $\mathsf{C}^{op}$ and $f\in\Hom_{\mathsf{C}^{op}} (A,B)=\Hom_\mathsf{C}(B,A)$, $g\in\Hom_{\mathsf{C}^{op}} (B,C)=\Hom_\mathsf{C}(C,B)$, the composition laws in $\mathsf{C}$ determines a morphism $f*g$ in $\Hom_{\mathsf{C}} (C,A)$, which deduces the composition defined on $\mathsf{C}^{op}$:
		\[
		\begin{aligned}
		\Hom_{\mathsf{C}^{op}} (A,B)\times\Hom_{\mathsf{C}^{op}} (B,C)&\longrightarrow \Hom_{\mathsf{C}^{op}} (A,C)\\
		(f,g)&\longmapsto g\circ f:=f*g
		\end{aligned}
		\]
		\item Associativity. If $f\in\Hom_{\mathsf{C}^{op}} (A,B)$, $g\in\Hom_{\mathsf{C}^{op}} (B,C)$, $h\in\Hom_{\mathsf{C}^{op}} (C,D)$, then
		\[
		f\circ(g\circ h)=f\circ(h*g)=(h*g)*f=h*(g*f)=(g*f)\circ h=(f\circ g)\circ h.
		\]
		\item Identity. For all $f\in\Hom_{\mathsf{C}^{op}} (A,B)$, we have
		\[
		f\circ 1_A=1_A*f=f,\quad 1_B\circ f=f*1_B=f.
		\]
	\end{itemize}
	Thus we get the full construction of $\mathsf{C}^{op}$.
\end{solution}

\subsection{\textsection4. Morphisms}
\begin{problem}[4.2]
	In Example 3.3 we have seen how to construct a category from a set endowed	with a relation, provided this latter is reflexive and transitive. For what types of relations is the corresponding category a groupoid (cf. Example 4.6)? [\textsection4.1]
\end{problem}
\begin{solution}
    For a reflexive and transitive relation $\sim$ on a set $S$, define the category $\mathsf{C}$ as follows:
    \begin{itemize}
    	\item Objects: $\mathrm{Obj}(\mathsf{C})=S$;
    	\item Morphisms: if $a, b$ are objects (that is: if $a, b \in S$) then let 
    	\[
    	\mathrm{Hom}_\mathsf{C}(a, b)=
    	\left\{
    	\begin{aligned}
    	&(a, b)\in S\times S &\text{ if } a\sim b\\	
    	& \emptyset &\text{ otherwise}\\
    	\end{aligned}
    	\right.
    	\]
    \end{itemize}
    In Example 3.3 we have shown the category. If the relation $\sim$ is endowed with symmetry, we have
    \[
    (a,b)\in\mathrm{Hom}_\mathsf{C}(a, b)\implies a\sim b\implies b\sim a\implies (b,a)\in\mathrm{Hom}_\mathsf{C}(b, a).
    \]
    Since
    \[
    (a,b)(b, a)=(a,a)=1_a,\quad(b, a)(a,b)=(b,b)=1_b,
    \]
    in fact $(a,b)$ is an isomorphism. From the arbitrariness of the choice of $(a,b)$, we show that $\mathsf{C}$ is a groupoid. Conversely, if $\mathsf{C}$ is a groupoid, we can show the relation $\sim$ is symmetric. To sum up, the category $\mathsf{C}$ is a groupoid
    if and only if the corresponding relation $\sim$ is an equivalence relation.
\end{solution}
~\\ 

\subsection{\textsection5. Universal properties}
\begin{problem}[5.1]
	Prove that a final object in a category $\mathsf{C}$ is initial in the opposite category $\mathsf{C}_{op}$	(cf. Exercise 3.1).
\end{problem}
\begin{solution}
	An object $F$ of $\mathsf{C}$ is final in $\mathsf{C}$ if and only if
	\[
	\forall A \in \Obj(\mathsf{C}) : \Hom_\mathsf{C}(A,F) \text{ is a singleton.}
	\]
	That is equivalent to
	\[
	\forall A \in \Obj(\mathsf{C}_{op}) : \Hom_{\mathsf{C}_{op}}(F,A) \text{ is a singleton,}
	\]
	which means $F$ is initial in the opposite category $\mathsf{C}_{op}$.	
\end{solution}


\section{Chapter II.\quad Groups, first encounter}

\subsection{\textsection1. Definition of group}
\begin{problem}[1.1]
	Write a careful proof that every group is the group of isomorphisms of a groupoid. In particular, every group is the group of automorphisms of some object in some category.
\end{problem}
\begin{solution}
	Assume $G$ is a group. Define a category $\mathsf{C}$ as follows:
	\begin{itemize}
		\item Objects: $\mathrm{Obj}(\mathsf{C})=\{*\}$;
		\item Morphisms: $\mathrm{Hom}_\mathsf{C}(*,*)=\mathrm{End}_\mathsf{C}(*)=G$.
	\end{itemize}	
	The composition of homomorphism is corresponding to the multiplication between two elements in $G$. The identity morphism on $*$ is $1_*=e_G$, which satisfies for all $g\in \mathrm{Hom}_\mathsf{C}(*,*)$,
	\[
	ge_G=e_Gg=g,
	\]
	and
	\[
	gg^{-1}=e_G,\;g^{-1}g=e_G.
	\]
	Thus any homomorphism $g\in \mathrm{Hom}_\mathsf{C}(*,*)$ is an isomorphism and accordingly $\mathsf{C}$ is a groupoid.
\end{solution}
~\\

\begin{problem}[1.4]
	Suppose that $g^2 = e$ for all elements $g$ of a group $G$; prove that $G$ is commutative.
\end{problem}
\begin{solution}
	For all $a,b\in G$,
	\[
	abab=e\implies a(abab)b=ab\implies (aa)ba(bb)=ab\implies ba=ab.
	\]
\end{solution}
~\\

\subsection{\textsection2. Examples of groups}
\begin{problem}[2.1]
	One can associate an $n\times n$ matrix $M_\sigma$ with a permutation $\sigma \in S_n$, by
	letting the entry at $(i, \sigma(i))$ be 1, and letting all other entries be 0. For example,
	the matrix corresponding to the permutation
	\[
	\sigma=\left(
	\begin{matrix}	
	1 & 2 & 3\\	
    3 & 1 & 2	
	\end{matrix}
	\right)\in S_3
	\]
	would be 
	\[
	M_\sigma=\left(
	\begin{matrix}	
	0 & 0 & 1\\	
	1 & 0 & 0\\	
	0 & 1 & 0
	\end{matrix}
	\right)
	\]
	Prove that, with this notation,
	\[
	M_{\sigma\tau}=M_{\sigma}M_{\tau}
	\]
	for all $\sigma,\tau\in S_n$, where the product on the right is the ordinary product of matrices.
\end{problem}
\begin{solution}
	By introducing the Kronecker delta function	
    \[
    \delta _{{i,j}}=
    	\begin{cases}
    	0{\text{\quad if }}i\neq j,\\
    	1{\text{\quad if }}i=j,
    	\end{cases}
    \]
	the entry at $(i,j)$ of the matrix $M_{\sigma\tau}$ can be written as
	\[
	(M_{\sigma\tau})_{i,j}=\delta_{\tau(\sigma(i)),j}
	\]
	and the entry at $(i,j)$ of the matrix $M_{\sigma}M_{\tau}$ can be written as
	\[
	(M_{\sigma}M_{\tau})_{i,j}=\sum_{k=1}^{n}(M_{\sigma})_{i,k}(M_{\tau})_{k,j}=\sum_{k=1}^{n}\delta_{\sigma(i),k}\cdot\delta_{\tau(k),j}=\sum_{k=1}^{n}\delta_{\sigma(i),k}\cdot\delta_{k,\tau^{-1}(j)}=\delta_{\sigma(i),\tau^{-1}(j)},
	\]
	where the last but one equality holds by the fact 
	\[
	\tau(k)=j\iff k=\tau^{-1}(j).
	\]
	Noticing that
	\[
	\tau(\sigma(i))=j\iff \sigma(i)=\tau^{-1}(j),
	\]
    we see $M_{\sigma\tau}=M_{\sigma}M_{\tau}$
    for all $\sigma,\tau\in S_n$.
\end{solution}
~\\

\begin{problem}[2.2]
	Prove that if $d \le n$, then $S_n$ contains elements of order $d$.
\end{problem}
\begin{solution}
	The cyclic permutation 
	\[
	\sigma=(1\;2\;3\cdots d)
	\]
    is an element of order $d$ in $S_n$.
\end{solution}

~\\

\begin{problem}[2.3]
	For every positive integer $n$ find an element of order $n$ in $S_\mathbb{N}$.
\end{problem}
\begin{solution}
The cyclic permutation 
\[
\sigma=(1\;2\;3\cdots n)
\]
is an element of order $d$ in $S_n$.
\end{solution}

~\\

\begin{problem}[2.4]	
	Define a homomorphism $D_8 \rightarrow S_4$ by labeling vertices of a square, as we did
	for a triangle in \textsection2.2. List the 8 permutations in the image of this homomorphism.
\end{problem}
\begin{solution}
	The image of $n$ rotations under the homomorphism are
	\[
	\sigma_1=e_{D_8},\;\sigma_2=(1\;2\;3\;4),\;\sigma_3=(1\;3)(2\;4),\;\sigma_4=(1\;4\;3\;2).
	\]
	The image of $n$ reflections under the homomorphism are
	\[
	\sigma_5=(1\;3),\;\sigma_6=(2\;4),\;\sigma_7=(1\;2)(3\;4),\;\sigma_8=(1\;4)(3\;2).
	\]
\end{solution}

~\\

\begin{problem}[2.11]	
	Prove that the square of every odd integer is congruent to 1 modulo 8.
\end{problem}
\begin{solution}
	Given an odd integer $2k+1$, we have
	\[
	(2k+1)^2=4k(k+1)+1,
	\]
	where $k(k+1)$ is an even integer. So $(2k+1)^2\equiv1\mod 8$.
\end{solution}

~\\

\begin{problem}[2.12]	
	Prove that there are no integers $a, b, c$ such that $a^2+b^2=3c^2$. (Hint: studying the equation $[a]^2_4+[b]^2_4=3[c]^2_4$ in $\mathbb{Z}/4\mathbb{Z}$, show that $a, b, c$ would all have to be even. Letting $a=2k, b=2l,c=2m$, you would have $k^2+l^2=3m^2$. What's wrong with that?)
\end{problem}
\begin{solution}
	\[
	a^2+b^2=3c^2\implies [a]^2_4+[b]^2_4=3[c]_4^2.
	\]
	Noting that $[0]^2_4=[0]_4,[1]^2_4=[1]_4,[2]^2_4=[0]_4,[3]^2_4=[1]_4$, we see $[c]_4^2$ must be $[0]_4$ and so do $[a]_4^2$ and $[b]_4^2$. Hence  $[a]_4,[b]_4,[b]_4$ can only be $[0]_4$ or $[2]_4$, which justifies letting $a=2k_1, b=2l_2,c=2m_1$. After substitution we have $k^2+l^2=3m^2$. Repeating this process $n$ times yields $a=2^nk_n, b=2^nl_n,c=2^nm_n$. For a sufficiently large number $N$, the absolute value of $k_N,l_N,m_N$ must be less than 1. Thus we conclude that $a=b=c=0$ is the unique solution to the equation $a^2+b^2=3c^2$.
\end{solution}
    
~\\

\begin{problem}[2.13]	
	Prove that if $\gcd(m, n) = 1$, then there exist integers $a$ and $b$ such that $am + bn = 1$. (Use Corollary 2.5.) Conversely, prove that if $am+ bn = 1$ for some integers $a$ and $b$, then $\gcd(m, n) = 1$. [2.15, \textsection V.2.1, V.2.4]
\end{problem}
\begin{solution}
	Applying corollary 2.5, we have $\gcd(m, n) = 1$ if and only if $[m]_n$ generates $\mathbb{Z}/n\mathbb{Z}$. Hence
	\[
	\gcd(m, n) = 1\iff a[m]_n=[1]_n\iff [am]_n=[1]_n\iff am+ bn = 1.
	\]
\end{solution}
    
~\\

\begin{problem}[2.15]	
	Let $n > 0$ be an odd integer.
	\begin{itemize}
		\item Prove that if $\gcd(m, n) = 1$, then $\gcd(2m+ n, 2n) = 1$. (Use Exercise 2.13.)
		\item Prove that if $\gcd(r, 2n) = 1$, then $\gcd(\frac{r+n}{2}, n) = 1$. (Ditto.)		
		\item Conclude that the function $[m]_n\rightarrow[2m + n]_{2n}$ is a bijection between $(\mathbb{Z}/n\mathbb{Z})^*$ and $(\mathbb{Z}/2n\mathbb{Z})^*$. 
	\end{itemize}
	The number $\phi(n)$ of elements of $(\mathbb{Z}/n\mathbb{Z})^*$ is Euler’s $\phi(n)$-function. The reader has just	proved that if $n$ is odd, then $\phi(2n) = \phi(n)$. Much more general formulas will be given later on (cf. Exercise V.6.8). [VII.5.11]
\end{problem}
\begin{solution}
	\begin{itemize}
		\item According to Exercise 2.13,
		\[
		\gcd(m, n) = 1\implies am+bn=1\implies \frac{a}{2}(2m+n)+\left(b-\frac{a}{2}\right)n=1.
		\]
		If $a$ is even, we have shown $\gcd(2m+ n, 2n) = 1$. Otherwise we can let $a'=a+n$ be an even integer and $b'=b-m$. Then it holds that
		\[
		\frac{a'}{2}(2m+n)+\left(b'-\frac{a'}{2}\right)n=1,
		\]
		which also indicates $\gcd(2m+ n, 2n) = 1$.
		\item If $\gcd(r, 2n) = 1$, then $r$ must be an odd integer and accordingly
		\[
		\gcd(2r+2n, 4n) = 1\implies a(2r+2n)+b(4n)=1\implies 4a \frac{r+n}{2}+4bn=1,
		\]
		which is $\gcd(\frac{r+n}{2}, n) = 1$.
		\item It is easy to check that the function $f:(\mathbb{Z}/n\mathbb{Z})^*\rightarrow(\mathbb{Z}/2n\mathbb{Z})^*,\;[m]_n\mapsto[2m + n]_{2n}$ is well-defined. The fact
		\[
		\begin{aligned}
		f([m_1]_n)=f([m_2]_n)&\implies
		f([2m_1 + n]_{2n})=f([2m_2 + n]_{2n})\\
		&\implies (2m_1 + n)-(2m_2 + n)=2kn\\
		&\implies m_1-m_2=kn\\
		&\implies [m_1]_n=[m_2]_n
		\end{aligned}
		\]
		indicates that $f$ is injective. For any $[r]_{2n}\in(\mathbb{Z}/2n\mathbb{Z})^*$, we have
		\[
		\gcd(r, 2n) = 1\implies\gcd\left(\frac{r+n}{2},n\right) = 1\implies \left[\frac{r+n}{2}\right]_n\in(\mathbb{Z}/n\mathbb{Z})^*,
		\]
		and
		\[
		f\left(\left[\frac{r+n}{2}\right]_n\right)=[r+2n]_{2n}=[r]_{2n},
		\]
		which indicates that $f$ is surjective. Thus we show $f$ is a bijection.
	\end{itemize}
\end{solution}

~\\

\begin{problem}[2.16]	
	Find the last digit of $1238237^{18238456}$. (Work in $\mathbb{Z}/10\mathbb{Z}$.)
\end{problem}
\begin{solution}
	\[
	1238237^{18238456}\equiv7^{18238456}\equiv(7^4)^{4559614}\equiv2401^{4559614} \equiv 1 \mod 10,
	\]
	which indicates that the last digit of $1238237^{18238456}$ is 1.
\end{solution}


~\\

\begin{problem}[2.17]	
	Show that if $m\equiv m'\mod n$, then $\gcd(m,n) = 1$ if and only if $\gcd(m',n)=1$. [\textsection 2.3]
\end{problem}
\begin{solution}
	Assume that $m-m'=kn$. If $\gcd(m,n) = 1$, for any common divisor $d$ of $m'$ and $n$
	\[
	d|m',\;d|n\implies d|(m'+kn)\implies d|m\implies d=1,
	\]
	which means $\gcd(m',n)=1$. Likewise, we can show $\gcd(m',n) = 1\implies\gcd(m,n)=1$
\end{solution}

~\\
\subsection{\textsection3. The category $\mathsf{Grp}$} 

\begin{problem}[3.1]	
	Let $\varphi : G\rightarrow H$ be a morphism in a category $\mathsf{C}$ with products. Explain why
	there is a unique morphism
	\[
	(\varphi\times\varphi) : G \times G \longrightarrow H \times H .
	\]
	(This morphism is defined explicitly for $\mathsf{C} = \mathsf{Set}$ in \textsection 3.1.)
\end{problem}
\begin{solution}
	By the universal property of product in $\mathsf{C}$, there exist a unique morphism $(\varphi\times\varphi) : G \times G \longrightarrow H \times H$ such that the following diagram commutes.
    \[\xymatrix{
    	G\ar[r]^{\varphi} & H \\
    	G \times G\ar[u]^{\pi_G}\ar[d]_{\pi_G}\ar[r]^{\varphi\times\varphi} &  H\times H\ar[u]_{\pi_H}\ar[d]^{\pi_H} \\
    	G\ar[r]^{\varphi} & H 
    }\]
\end{solution}

~\\

\begin{problem}[3.2]	
	Let $\varphi : G\rightarrow H, \psi : H \rightarrow K$ be morphisms in a category with products, and
	consider morphisms between the products $G\times G, H\times H, K\times K$ as in Exercise 3.1.
	Prove that
	\[
	(\psi\varphi) \times(\psi\varphi)=(\psi \times \psi)(\varphi\times \varphi) .
	\]
	(This is part of the commutativity of the diagram displayed in \textsection 3.2.)
\end{problem}
\begin{solution}
	By the universal property of product in $\mathsf{C}$, there exist a unique morphism 
	\[
	(\psi\varphi) \times(\psi\varphi):G\times G\rightarrow K\times K
	\] 
	such that the following diagram commutes.
	\[\xymatrix{
		G\ar[rr]^{\psi\varphi} && H \\
		G \times G\ar[u]^{\pi_G}\ar[d]_{\pi_G}\ar[rr]^{	(\psi\varphi) \times(\psi\varphi)} &&  H\times H\ar[u]_{\pi_H}\ar[d]^{\pi_H} \\
		G\ar[rr]^{\psi\varphi} && H 
	}\]
    As the following commuting diagram tells us the composition 
    \[
    (\psi \times \psi)(\varphi\times \varphi):G\times G\rightarrow K\times K
    \]
    can make the above diagram commute,
	\[\xymatrix{
		G\ar[r]^{\varphi}\ar@/^1.6pc/[rr]^{\psi\varphi} & H\ar[r]^{\psi} & K \\
		G \times G\ar[u]^{\pi_G}\ar[d]_{\pi_G}\ar[r]^{\varphi\times\varphi} &  H\times H\ar[u]^{\pi_H}\ar[d]_{\pi_H}\ar[r]^{\psi\times\psi} &  K\times K \ar[u]^{\pi_K}\ar[d]_{\pi_K}\\
		G\ar[r]^{\varphi}\ar@/_1.6pc/[rr]_{\psi\varphi} & H \ar[r]^{\psi} & K
	}\]
    there must be $(\psi\varphi) \times(\psi\varphi)=(\psi \times \psi)(\varphi\times \varphi)$.
    
\end{solution}

~\\

\begin{problem}[3.3]	
	Show that if $G, H$ are abelian groups, then $G \times H$ satisfies the universal property for coproducts in $\mathsf{Ab}$.
\end{problem}
\begin{solution}
	Define two monomorphisms:
	\[
	i_G:G\longrightarrow G\times H,\;a\longmapsto (a,0_H)
	\]
	\[
	i_H:H\longrightarrow G\times H,\;b\longmapsto (0_G,b)
	\]
	We are proving that for any two homomorphisms $g:G\rightarrow M$ and $h:H\rightarrow M$ in $\mathsf{Ab}$, the map
	\[
	\begin{aligned}
	\varphi:\quad & G\times H\longrightarrow M,\\
	         & (a,b)\longmapsto g(a)+h(b)
	\end{aligned}
    \]
    is a homomorphism and makes the following diagram commute. 
	\[\xymatrix{
		G\ar[rd]^{g}\ar[d]_{i_G}  \\
		G \times H\ar[r]^{\varphi} &  M\\
		H\ar[ru]_{h}\ar[u]^{i_H}  
	}\]
	Exploiting the fact that $g,h$ are homomorphisms and $M$ is an abelian group, it is easy to check that $\varphi$ preserves the addition operation
	\[
	\begin{aligned}
	\varphi((a_1,b_1)+(a_2,b_2))&=\varphi((a_1+a_2,b_1+b_2))\\
	&=g(a_1+a_2)+h(b_1+b_2)\\
	&=(g(a_1)+g(a_2))+(h(b_1)+h(b_2))\\
	&=(g(a_1)+h(b_1))+(g(a_2)+h(b_2))\\
	&=g(a_1+b_1)+h(a_2+b_2)\\
	&=\varphi((a_1,b_1))+\varphi((a_2,b_2))
	\end{aligned}
	\]
	and the diagram commutes
	\[
	\varphi\circ i_G(a)=\varphi((a,0_H))=g(a)+h(0_H)=g(a)+0_M=g(a),
	\]
	\[
	\varphi\circ i_H(b)=\varphi((0_G,b))=g(0_G)+h(b)=0_M+h(b)=h(b).
	\]
	To show the uniqueness of the homomorphism $\varphi$ we have constructed, suppose a homomorphism $\varphi'$ can make the diagram commute. Then we have
	\[
	\varphi'((a,b))=\varphi'((a,0_H)+(0_G,b))=\varphi'(i_G(a))+\varphi'(i_H(b))=g(a)+h(b)=\varphi((a,b)),
	\]
	that is $\varphi'=\varphi$. Hence we show that there exist a unique homomorphism $\varphi$ such that the diagram commutes, which amounts to the universal property for coproducts in $\mathsf{Ab}$.
	
\end{solution}

~\\

\begin{problem}[3.4]	
	Let $G, H$ be groups, and assume that $G\cong H\times G$. Can you conclude that $H$ is trivial? (Hint: No. Can you construct a counterexample?)
\end{problem}
\begin{solution}
	Consider the function
	\[
	\begin{aligned}
	\varphi:\;&\mathbb{Z}\times\mathbb{Z}[x]\longrightarrow\mathbb{Z}[x]\\  
	  &(n,f(x))\longmapsto n+xf(x)
	\end{aligned}
	\]
	Firstly, we can show $\varphi$ is a homomorphism as follows
	\[
	\begin{aligned}
	\varphi((n_1,f_1(x))+(n_2,f_2(x)))&=\varphi((n_1+n_2,f_1(x)+f_2(x)))\\
	&=(n_1+n_2)+x(f_1(x)+f_2(x))\\
	&=(n_1+xf_1(x))+(n_2+xf_2(x))\\
	&=\varphi((n_1,f_1(x)))+\varphi((n_2,f_2(x))).
	\end{aligned}
	\]
	Secondly, we are to show $\varphi$ is a monomorphism. It follows by
	\[
	\varphi((n,f(x)))=n+xf(x)=0\implies n=0,\;f(x)=0\implies \ker\varphi=\{(0,0)\}.
	\]
	Lastly, since the cardinal numbers of both $\mathbb{Z}\times\mathbb{Z}[x]$ and $\mathbb{Z}[x]$ are $\aleph_0$, $\varphi$ is indeed an isomorphism. Therefore, as a counterexample we have $\mathbb{Z}[x]\cong\mathbb{Z}\times\mathbb{Z}[x]$.
\end{solution}

~\\

\begin{problem}[3.5]	
	Prove that $\mathbb{Q}$ is not the direct product of two nontrivial groups.
\end{problem}
\begin{solution}
    Consider the additive group of rationals $(\mathbb{Q},+)$. Assume that $\varphi$ is a isomorphism between the product $G\times H=\{(a,b)|a\in G,b\in H\}$ and $(\mathbb{Q},+)$. Note that $\{e_G\}\times H$ and $G\times \{e_H\}$ are subgroups in $G\times H$ and their intersection is the trivial group $\{(e_G,e_H)\}$. It is easy to check that bijection $\varphi$ satisfies $\varphi(A\cap B)=\varphi(A)\cap\varphi (B)$. So applying the fact we have
    \[
    \varphi(\{(e_G,e_H)\})=\varphi(\{e_G\}\times H\cap G\times \{e_H\})=\varphi(\{e_G\}\times H)\cap \varphi(G\times \{e_H\})=\{0\}.
    \] 
    Suppose both $\varphi(\{e_G\}\times H)$ and $\varphi(G\times \{e_H\})$ are nontrivial groups. If $\dfrac{p}{q}\in \varphi(\{e_G\}\times H)-\{0\}$ and $\dfrac{r}{s}\in \varphi(G\times \{e_H\})-\{0\}$, there must be 
    \[
    rp=rq\cdot\dfrac{p}{q}=ps\cdot\dfrac{r}{s}\in \varphi(\{e_G\}\times H)\cap\varphi(G\times \{e_H\}),
    \]
    which implies $rp=0$. Since both $\dfrac{p}{q}$ and $\dfrac{r}{s}$ are non-zero, it leads to a contradiction. Thus without loss of generality we can assume $\varphi(\{e_G\}\times H)$ is a trivial group $\{0\}$. 
    Since $\varphi$ is isomorphism, we see that for all $h\in H$,
    \[
    \varphi(e_G,h)=\varphi(e_G,e_H)=0\iff h=e_H.
    \]
    That is, $H$ is a trivial group. Therefore, we have shown $(\mathbb{Q},+)$ will never be isomorphic to the direct product of two nontrivial groups.
\end{solution}

~\\

\begin{problem}[3.6]	
	Consider the product of the cyclic groups $C_2,C_3$ (cf. \textsection 2.3): $C_2 \times C_3$. By Exercise 3.3, this group is a coproduct of $C_2$ and $C_3$ in $\mathsf{Ab}$. Show that it is not a coproduct of $C_2$ and $C_3$ in $\mathsf{Grp}$, as follows:
	\begin{itemize}
		\item find injective homomorphisms $C_2\rightarrow S_3$, $C_3 \rightarrow S_3$;
		\item arguing by contradiction, assume that $C_2\times C_3$ is a coproduct of $C_2, C_3$, and deduce that there would be a group homomorphism $C_2\times C_3\rightarrow S_3$ with certain properties;
		\item show that there is no such homomorphism.
	\end{itemize}
\end{problem}
\begin{solution}
	\begin{itemize}
	\item 
	Monomorphisms $g:C_2\rightarrow S_3$, $h:C_3 \rightarrow S_3$ can be constructed as follows:
	\[
	g([0]_2)=e,g([1]_2)=\left(
	\begin{matrix}	
	1 & 2 & 3\\	
	1 & 3 & 2	
	\end{matrix}
	\right).
	\]
	\[
	h([0]_3)=e,h([1]_3)=\left(
	\begin{matrix}	
	1 & 2 & 3\\	
	3 & 1 & 2	
	\end{matrix}
	\right),h([2]_3)=\left(
	\begin{matrix}	
	1 & 2 & 3\\	
	2 & 3 & 1	
	\end{matrix}
	\right).
	\]
	\item Supposing that $C_2\times C_3$ is a coproduct of $C_2, C_3$, there would be a unique group homomorphism $\varphi :C_2\times C_3\rightarrow S_3$ such that the following diagram commutes
	\[\xymatrix{
		C_2\ar[rd]^{g}\ar[d]_{i_{C_2}}  \\
		C_2 \times C_3\ar[r]^{\varphi} &  S_3\\
		C_3\ar[ru]_{h}\ar[u]^{i_{C_3}}  
	}\]
    In other words, for all $a\in C_2,b\in C_3$,
    \[
    \begin{aligned}
    \varphi(a,b)&=\varphi(([0]_2,b)+(a,[0]_3))=\varphi(([0]_2,b))\varphi((a,[0]_3))=\varphi(i_{C_3}(b))\varphi(i_{C_2}(a))=h(b)g(a)\\
    &=\varphi((a,[0]_3)+([0]_2,b))=\varphi((a,[0]_3))\varphi(([0]_2,b))=\varphi(i_{C_2}(a))\varphi(i_{C_3}(b))=g(a)h(b).
    \end{aligned}
    \]
	\item Since
	\[
	g([1]_2)h([1]_3)=\left(
	\begin{matrix}	
	1 & 2 & 3\\	
	1 & 3 & 2	
	\end{matrix}
	\right)
	\left(
	\begin{matrix}	
	1 & 2 & 3\\	
	3 & 1 & 2	
	\end{matrix}
	\right)=
	\left(
	\begin{matrix}	
	1 & 2 & 3\\	
	3 & 2 & 1	
	\end{matrix}
	\right),
	\]
	\[
	h([1]_3)g([1]_2)=
	\left(
	\begin{matrix}	
	1 & 2 & 3\\	
	3 & 1 & 2	
	\end{matrix}
	\right)
	\left(
	\begin{matrix}	
	1 & 2 & 3\\	
	1 & 3 & 2	
	\end{matrix}
	\right)
	=
	\left(
	\begin{matrix}	
	1 & 2 & 3\\	
	2 & 1 & 3	
	\end{matrix}
	\right),
	\]
	we see $g(a)h(b)\ne h(b)g(a)$ not always holds. The derived contradiction shows that $C_2\times C_3$ is not a coproduct of $C_2, C_3$ in $\mathsf{Grp}$. 
\end{itemize}	
\end{solution}

\begin{problem}[3.7]	
	Show that there is a surjective homomorphism $Z*Z\rightarrow C_2*C_3$. ($*$ denotes coproduct in $\mathsf{Grp}$.)
	
\end{problem}

\begin{solution}
	Consider the mapping
	\[
	\begin{aligned}
	\varphi:\;&\mathbb{Z}*\mathbb{Z}\longrightarrow C_2*C_3\\  
	&x^{m_1}y^{n_1}\cdots x^{m_k}y^{n_k}\longmapsto x^{[m_1]_2}y^{[n_1]_3}\cdots x^{[m_k]_2}y^{[n_k]_3}
	\end{aligned}
	\]
	Since
	\[
	\begin{aligned}
	&\varphi(x^{m_1}y^{n_1}\cdots x^{m_k}y^{n_k}x^{m_1'}y^{n_1'}\cdots x^{m_{k'}'}y^{n_{k}'})\\
	=&x^{[m_1]_2}y^{[n_1]_3}\cdots x^{[m_k]_2}y^{[n_k]_3}x^{[m_1']_2}y^{[n_1']_3}\cdots x^{[m_k']_2}y^{[n_k']_3}\\
	=&\varphi(x^{m_1}y^{n_1}\cdots x^{m_k}y^{n_k})\varphi(x^{m_1'}y^{n_1'}\cdots x^{m_{k'}'}y^{n_{k}'})
	\end{aligned},
	\]
	$\varphi$ is a homomorphism. It is clear that $\varphi$ is surjective. Thus we show there exists a surjective homomorphism $Z*Z\rightarrow C_2*C_3$.
\end{solution}
~\\
\begin{problem}[3.8]	
	Define a group $G$ with two generators $x, y$, subject (only) to the relations $x^2 = e_G, y^3 = e_G$. Prove that $G$ is a coproduct of $C_2$ and $C_3$ in $\Grp$. (The reader	will obtain an even more concrete description for $C_2*C_3$ in Exercise 9.14; it is called the modular group.) [\textsection3.4, 9.14]
	
\end{problem}

\begin{solution}
Given the maps $i_1:C_2\rightarrow G,[m]_2\mapsto x^m$ and $i_2:C_3\rightarrow G,[n]_3\mapsto y^n$, we can check that $i_1$, $i_2$ are homomorphisms. 
We are to show that for every group $H$ endowed with two homomorphisms $f_1:C_2\rightarrow H$, $f_2:C_3\rightarrow H$ , there would be a unique group homomorphism $\varphi :G\rightarrow H$ such that the following diagram commutes
\[\xymatrix{
	C_2\ar[rd]^{f_1}\ar[d]_{i_1}  \\
	G\ar[r]^{\varphi} &  H\\
	C_3\ar[ru]_{f_2}\ar[u]^{i_2}  
}\]	
or
\[
\varphi(i_1([m]_2))=\varphi(x^m)=\varphi(x)^m=f_1([m]_2),
\]
\[
\varphi(i_2([n]_3))=\varphi(y^n)=\varphi(y)^n=f_2([n]_3).
\]
Define $\phi:G\rightarrow H$ as $\phi(x^my^n)=f_1([m]_2)f_2([n]_3)$, $\phi(y^nx^m)=f_2([n]_3)f_1([m]_2)$. It is clear to see $\phi$ makes the diagram commute. Moreover, if $\varphi$ makes the diagram commute, it follows that for all $x^my^n,y^nx^m\in G$,
\[
\varphi(x^my^n)=\varphi(x^m)\varphi(y^n)=f_1([m]_2)f_2([n]_3),
\]
\[
\varphi(y^nx^m)=\varphi(y^n)\varphi(x^m)=f_2([n]_3)f_1([m]_2),
\]
which implies $\varphi=\phi$. Thus we can conclude $G$ is the coproduct of $C_2$ and $C_3$ in $\Grp$.

\end{solution}



~\\

\end{document}
